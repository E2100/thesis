\section{Conclusion}
\label{sec:conclusion}

The information overload problem will always be present.
No matter how elegant solutions one may find,
the fact is that the overwhelming amount of available data
quickly outgrows our ability to use it.
We believe artificial intelligence is crucial to finding a solution.
Only by creating intelligent systems that 
help us filter, sort and consume information can we hope 
to mitigate the overload.

As stated through the latent subjectivity problem,
systems should not only tell users what has been predicted,
but also allow flexible and adaptive usage of its internal algorithms.
\emph{After all, a system that insists on being adaptive
in one particular way is not really adaptive at all}.


\section*{Acknowledgments}
\label{sec:method}

This paper is a short version of my Master Thesis in Artificial Intelligence
\cite{Bjorkoy2011}.
This thesis can be accessed 
online\footnote{See \url{github.com/olav/thesis/raw/master/thesis/dist/thesis.pdf}.},
and provides more background theory, 
experiments and results.

I would like to thank my supervisor, assistant professor Asbjørn Thomassen, for valuable guidance and feedback throughout the process.
In addition, thanks are in order for my fellow students 
Kim Joar Bekkelund and Kjetil Valle,
who helped me formulate my thoughts and provided feedback on the work represented by this paper.


