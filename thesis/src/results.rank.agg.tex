\section{Rank Aggregation}

\begin{table}[h]
  \centering 

  \begin{tabular*}{0.9\textwidth}{ r l p{8cm} }
    \multicolumn{3}{l}{Results and their IR ranking score for the query [\emph{``new york'' or washington}]:}\\
    \toprule
    \emph{\#} & \emph{score} & \emph{title}\\
    \midrule
    1 & 2.8419  &  New York Cop (1996)                    \\
    2 & 2.8419  &  King of New York (1990)                \\
    3 & 2.8419  &  Autumn in New York (2000)              \\
    4 & 2.8419  &  Couch in New York                      \\
    5 & 2.4866  &  Escape from New York (1981)            \\
    6 & 2.4866  &  All the Vermeers in New York (1990)    \\
    7 & 2.1314  &  Home Alone 2: Lost in New York (1992)  \\
    8 & 1.0076  &  Saint of Fort Washington               \\
    9 & 1.0076  &  Washington Square (1997)               \\
    10& 0.8816  &  Mr. Smith Goes to Washington (1939)    \\
    \bottomrule
  \end{tabular*}

  \vspace{1em} 

  \begin{tabular*}{0.9\textwidth}{ r l p{8.5cm} l }
    \multicolumn{4}{l}{Predicted ratings from the adaptive recommenders method for each item:}\\
    \toprule
    \emph{\#} & \emph{score} & \emph{title} & $\Delta_{IR}$\\
    \midrule
    1 & 3.7255  &  Mr. Smith Goes to Washington (1939)    & \color{green} $\uparrow$ 9 \\
    2 & 3.1430  &  Escape from New York (1981)            & \color{green} $\uparrow$ 3 \\
    3 & 3.0003  &  King of New York (1990)                & \color{red} $\downarrow$ 1 \\
    4 & 2.9498  &  Washington Square (1997)               & \color{green} $\uparrow$ 5 \\
    5 & 2.7258  &  Saint of Fort Washington               & \color{green} $\uparrow$ 3 \\
    6 & 2.6862  &  Couch in New York                      & \color{red} $\downarrow$ 2 \\
    7 & 2.6380  &  All the Vermeers in New York (1990)    & \color{red} $\downarrow$ 1 \\
    8 & 2.1601  &  Home Alone 2: Lost in New York (1992)  & \color{red} $\downarrow$ 1 \\
    9 & 1.7241  &  Autumn in New York (2000)              & \color{red} $\downarrow$ 6 \\
    10& 0.0     &  New York Cop (1996)                    & \color{red} $\downarrow$ 6 \\
    \bottomrule
  \end{tabular*}

  \vspace{1em} 

  \begin{tabular*}{0.9\textwidth}{ r l p{8.5cm} l }
    \multicolumn{4}{l}{Final results list with IR and adaptive predictions combined:}\\
    \toprule
    \emph{\#} & \emph{score} & \emph{title} & $\Delta_{IR}$ \\
    \midrule
    1 & 5.8422  &  King of New York (1990)                & \color{green} $\uparrow$ 1 \\
    2 & 5.6297  &  Escape from New York (1981)            & \color{green} $\uparrow$ 3 \\
    3 & 5.5281  &  Couch in New York                      & \color{green} $\uparrow$ 1 \\
    4 & 5.1247  &  All the Vermeers in New York (1990)    & \color{green} $\uparrow$ 2 \\
    5 & 4.6072  &  Mr. Smith Goes to Washington (1939)    & \color{green} $\uparrow$ 5 \\
    6 & 4.5661  &  Autumn in New York (2000)              & \color{red} $\downarrow$ 3 \\
    7 & 4.2915  &  Home Alone 2: Lost in New York (1992)  & \color{black} $=$ \\
    8 & 3.9575  &  Washington Square (1997)               & \color{green} $\uparrow$ 1 \\
    9 & 3.7334  &  Saint of Fort Washington               & \color{red} $\downarrow$ 1 \\
    10& 2.8419  &  New York Cop (1996)                    & \color{red} $\downarrow$ 9 \\
    \bottomrule
  \end{tabular*}

  \vspace{1em}
  \caption[Complex IR Query]{
    Complex IR query:
    The first table shows the results returned by our IR model, defining the item-space for the following tables.
    The middle table shows the predicted ratings for each of the items in the results set.
    $\Delta_{IR}$ shows how much each item has moved compared to the initial IR results.
    Notably, the recommenders were not able to predict the rating for the movie "new york cop",
    which results in a low final placement for this item.
  }
  \label{table:rank:washington}
\end{table}


\afterpage{\clearpage}

While the previous experiment was a quantitative exploration of RMSE values,
this experiment will focus on qualitative traits of rank aggregation.
In particular we wish to see how our method can perform personalized search.

The MovieLens dataset fit the needs of this experiment.
Searching through movies is a scenario where the actual predicted
rating of each movie would be a welcome signal for ranking the results.
In other words, we have a database of movies the user wishes to search through,
where the results are ranked both by how well they match the free text query,
and according to the predicted rating of each movie for the current user.

Hypothesis H3 states that 
{
  \itshape
  the result set from an information retrieval query
  can be personalized with adaptive recommenders.
}
We wish to check if the prediction algorithm
in Listing \ref{code:rank:prediction} performs personalized search
in a meaningful way.
There are a few important limitations to this experiment:

\begin{itemize}
  \item 
    We are not interested in measuring the actual performance of the IR system.
    It us assumed that the IR model returns items relevant to the current query,
    ranked by their individual relevance.
  \item
    We are not interested in measuring the performance of the resulting personalized search.
    This experiment will only show whether or not personalized search is achievable
    by using adaptive recommenders, as per our hypothesis.
\end{itemize}

\begin{comment}
\begin{table}[b]
  \centering
  \begin{tabular*}{0.7\textwidth}{ l l l l }
    \toprule
      ~ & 
      \emph{query} &
      \emph{scores} &
      \emph{IR weight} \\
    \midrule
    
    1 &
    [``new york or washington''] &
    combined &
    $1.0$ \\

    2 &
    [star trek] &
    combined &
    $0.3$ \\

    3 &
    [paris] &
    ratings &
    $0.0$ \\

    4 &
    [1998] &
    ratings &
    $0.0$ \\

    \bottomrule 
  \end{tabular*}
  \caption[List of Ranking Experiments]{List of ranking tests in this section.}
  \label{table:experiments:rank}
\end{table}
\end{comment}

In other words, this experiment is not statistically significant in any way,
but is should show how our approach to recommender systems
can be used to provide personalized search.

There are many ways of determining the accuracy of a personalized search
algorithm, such as the mean average precision of the results list
(the mean of the precision average over a set of queries).
However, these are subjective measures based on relevance judgements from each user.
Our hypothesis only states that our algorithm should be usable for such 
a system, which is what we shall explore in this section.
To quantitatively measure the performance of personalized search,
one would need detailed query logs, user profiles and click-through information,
which we have no access to.

Of course, any recommender system can be used for personalized search.
The interesting bit in regards to adaptive recommenders is what 
happens under the hood. First, the information retrieval score 
is itself treated as an input signal, just as the user modeling methods.
Second, by using adaptive recommenders, the user is in control of which
methods that actually determine how the final results are ranked.
We shall take a look at four use cases with to see how our algorithm
performs in a number of scenarios. Each case presents 
a query and shows how a certain IR weight influences the final ranking.

The IR weight is the ratio multiplied with the IR score 
before the adaptive recommender scores are incorporated
(see Listing \ref{code:rank:prediction}).
The actual choice of weight depends on scale of scores
returned by the IR method.
If the scores are on the same scales as the ratings themselves,
an IR weight of $1$ signifies that the IR score
should have equal importance as each recommender score.
Any higher, and the IR model should be prioritized above the recommenders.
Any lower, and the recommender scores will dominate the initial IR rankings.

In other words, the actual IR weight must
be calculated based on the scale of the IR scores.
In this chapter, the scores returned by our IR
model is normalized to the scale of our ratings (1-5).

We will consider the following use cases:
(1) searching multiple topics,
(2) searching for a series of movies,
(3) searching for one particular topic, and
(4) searching for a particular attribute.

\begin{table}[h]
  \centering 

  \begin{tabular*}{0.9\textwidth}{ r l p{8cm} }
    \multicolumn{3}{l}{Results and their IR ranking score for the query \emph{star trek}:}\\
    \toprule
    \emph{\#} & \emph{score} & \emph{title}\\
    \midrule
    1 & 4.2288 & Star Trek: Generations                 \\
    2 & 3.7002 & Star Trek: First Contact               \\
    3 & 3.7002 & Star Trek: The Wrath of Khan           \\
    4 & 3.7002 & Star Trek: The Motion Picture          \\
    5 & 3.1716 & Star Trek VI: The Undiscovered Country \\
    6 & 3.1716 & Star Trek III: The Search for Spock    \\
    7 & 3.1716 & Star Trek IV: The Voyage Home          \\
    8 & 3.1716 & Star Trek V: The Final Frontier        \\
    9 & 0.9670 & Star Wars                              \\
    10& 0.9670 & Lone Star                              \\
    \bottomrule
  \end{tabular*}

  \vspace{1em} 

  \begin{tabular*}{0.9\textwidth}{ r l p{8.5cm} l }
    \multicolumn{4}{l}{Predicted ratings from the stacked recommender method for each item:}\\
    \toprule
    \emph{\#} & \emph{score} & \emph{title} & $\Delta_{IR}$\\
    \midrule
    1 & 4.8232 & Star Wars                              & \color{green} $\uparrow$ 9 \\
    2 & 4.6016 & Lone Star                              & \color{green} $\uparrow$ 8 \\
    3 & 4.2192 & Star Trek: The Wrath of Khan           & \color{black} $=$ \\
    4 & 4.0324 & Star Trek: First Contact               & \color{red} $\downarrow$ 2 \\
    5 & 3.8667 & Star Trek: Generations                 & \color{red} $\downarrow$ 4 \\
    6 & 3.7100 & Star Trek IV: The Voyage Home          & \color{green} $\uparrow$ 1 \\
    7 & 3.5604 & Star Trek VI: The Undiscovered Country & \color{red} $\downarrow$ 2 \\
    8 & 3.4420 & Star Trek: The Motion Picture          & \color{red} $\downarrow$ 4 \\
    9 & 3.4242 & Star Trek III: The Search for Spock    & \color{red} $\downarrow$ 3 \\
    10& 2.5249 & Star Trek V: The Final Frontier        & \color{red} $\downarrow$ 2 \\
    \bottomrule
  \end{tabular*}

  \vspace{1em} 

  \begin{tabular*}{0.9\textwidth}{ r l p{8.5cm} l }
    \multicolumn{4}{l}{Final results list with IR and stacked predictions combined:}\\
    \toprule
    \emph{\#} & \emph{score} & \emph{title} & $\Delta_{IR}$ \\
    \midrule
    1 & 5.5507  &    Star Trek: The Wrath of Khan            & \color{green} $\uparrow$ 2 \\
    2 & 5.5205  &    Star Trek: First Contact                & \color{black} $=$ \\
    3 & 5.3157  &    Star Trek: Generations                  & \color{red} $\downarrow$ 2 \\
    4 & 5.1187  &    Star Wars                               & \color{green} $\uparrow$ 5 \\
    5 & 4.9744  &    Star Trek IV: The Voyage Home           & \color{green} $\uparrow$ 2 \\
    6 & 4.7596  &    Star Trek III: The Search for Spock     & \color{black} $=$ \\
    7 & 4.7595  &    Star Trek: The Motion Picture           & \color{red} $\downarrow$ 3 \\
    8 & 4.7553  &    Star Trek VI: The Undiscovered Country  & \color{red} $\downarrow$ 3 \\
    9 & 4.6376  &    Lone Star                               & \color{green} $\uparrow$ 1 \\
    10& 4.0934  &    Star Trek V: The Final Frontier         & \color{red} $\downarrow$ 2 \\
    \bottomrule
  \end{tabular*}
  \vspace{1em}
  \caption[Adaptive Rank Rescoring]{
    These three table show adaptive rank rescoring for the query "star trek".
    In this example, an IR weight of $0.3$ was used, instructing the algorithm to
    put about the same confidence in the IR score and recommender scores.
    In other words, each score is considered an input signal, and each 
    signal is weighted the same.
  }
  \label{table:rank:startrek}
\end{table}


\afterpage{\clearpage}

(1) Let us start with a simple use case:
a user wishes to find movies about two separate topics, ranked by 
query match and predicted ratings.
This is a realistic use case, for example if a user is interested
in a few topics and wants to see the movie within these categories
he or she will probably like the most.
The IR algorithm takes care of finding the items within the categories,
while the adaptive recommenders finds the most enjoyable movies,
according to the metrics most preferred by this user in the past.

Table \ref{table:rank:washington} shows this use case and how our algorithm performs.
The results are for the query [\emph{``new york'' or washington}].
The first table shows the IR scores for the first 10 results,
and their rank (position in the list) according to these scores.
The second table shows the predicted rankings for each of these items.
Finally, the third table shows the ranking after the IR scores
and predicted ratings have been combined.
The final column shows how each item have moved in relation to the 
IR results list.

In this run of the algorithm, the IR weight ($w_{IR}$) was set to $1.0$,
instructing the algorithm place about the same importance on the IR score
and the predicted ratings. As we can see in the last section of 
Table \ref{table:rank:washington} the final result list is a blend
of the IR rankings and prediction rankings.
In other words, we have achieved personalized search. The results
from the IR method are re-ranked according to personal preferences.


(2) Let us consider another use case:
A user wishes to see a movie in a certain series of movies,
but does not know which one. In this case, the IR method can find all movies within this series,
while the recommender systems ranks the result list according to the user's preferences.

Table \ref{table:rank:startrek} shows the intermediary and final ranking
for the query [\emph{star trek}], which refers to a movie series.
The IR method returns all items that match this query,
and the recommenders predict the rating for each of these items.
However, since the IR method only ranks results based on how well they match the query,
and the recommenders only care about the predicted rating, the combined result
list can get the best of both worlds:
the top ranked items are the ones that both match the query \emph{and}
are probable good fits for the current user.

(3) What happens when the IR weight is set to $0$?
In this use case, the predicted ratings alone sort the final list.
Consider the following use case:
A user wishes to see a movie related to a certain topic, e.g. a city.
Table \ref{table:rank:paris} shows two results lists for the query [\emph{paris}].
On the left are the standard ranking as returned by the IR model for this query,
along with their respective scores.
On the right we see the same results, re-ranked by user preferences.

For simple one-word queries, ignoring the IR score seems to give us the desired effect.
When we can be sure that each item returned by a search have the same textual relevance
(IR score), the IR method does not have any more information on which to rank
the results. The ranking then becomes the task of the recommender systems.
By employing adaptive recommenders, the results are not only ranked by 
one or more recommenders as chosen by the system, but by those of the recommenders
best suited to the current user. At the same time, each of these recommenders
are used differently for each item in the list, based on how well they have
performed for each item in the past.

(4) As we can see, ignoring the IR score gives us quite a different algorithm.
Now, the search part is only performed to constrain the item-space worked
on by the recommender systems.
Another example of this is shown in Table \ref{table:rank:year}.
In this scenario, the user wishes to see a movie from a certain year,
and issues the query [\emph{1998}].
Naturally, the IR algorithm returns a whole lot of items, and each can
be said to be perfect answers to the algorithm -- each movie
was made in 1998.

In this case, setting the IR weight to $0$ allows us to rank the results
purely by predicted preference, which makes sense when the IR algorithm
can not rank the results in any meaningful way.
Note that the items in the left and right table are non-overlapping.
This is because only the first 10 results are shown.
The IR model returns a large number of items,
all with the same ranking score.
The recommender systems do the final ranking, and actually
push every item in the top 10 IR ranking 
below the top 10 final results.

\begin{table}[t]
  \centering 
  \begin{minipage}{0.49\textwidth}
    \centering 

  \begin{tabular*}{\textwidth}{ r l l }
    \toprule
    \emph{\#} & \emph{score} & \emph{title}\\
    \midrule
    1 & 3.0149  &  An American in Paris       \\
    2 & 3.0149  &  Paris Is Burning           \\
    3 & 3.0149  &  Paris - Texas              \\
    4 & 3.0149  &  Paris Was a Woman          \\
    5 & 3.0149  &  Forget Paris               \\
    6 & 3.0149  &  Window to Paris            \\
    7 & 3.0149  &  Jefferson in Paris         \\
    8 & 3.0149  &  Paris - France             \\
    9 & 2.6648  &  Rendezvous in Paris        \\
    10& 2.2611  &  Last Time I Saw Paris  \\
    \bottomrule
  \end{tabular*}
\end{minipage} 
\hfill 
\begin{minipage}{0.49\textwidth}
  \begin{tabular*}{\textwidth}{ r l l l }
    \toprule
    \emph{\#} & \emph{rating} & \emph{title} & $\Delta$\\
    \midrule
    1 & 3.5277 &  An American in Paris        & \color{black} $=$ \\
    2 & 3.3416 &  Forget Paris                & \color{green} $\uparrow$ 3 \\
    3 & 3.2037 &  Paris - Texas               & \color{black} $=$ \\
    4 & 3.1870 &  Window to Paris             & \color{green} $\uparrow$ 2 \\
    5 & 3.1409 &  Paris Is Burning            & \color{red} $\downarrow$ 3 \\
    6 & 3.1059 &  Last Time I Saw Paris       & \color{green} $\uparrow$ 4 \\
    7 & 2.7940 &  Rendezvous in Paris         & \color{green} $\uparrow$ 2 \\
    8 & 2.2964 &  Paris - France              & \color{black} $=$ \\
    9 & 1.7984 &  Jefferson in Paris          & \color{red} $\downarrow$ 2 \\
    10& 0.9420 &  Paris Was a Woman           & \color{red} $\downarrow$ 6 \\
    \bottomrule
  \end{tabular*}
  \end{minipage} 
  \vspace{1em}
  \caption[Completely Adaptive Ranking]{
    Completely adaptive ranking: With the IR weight set to $0$,
    the adaptive recommender is alone responsible for sorting the results.
    In this example, the IR model returns a list of items for the query "paris",
    and the adaptive user models sorts the results according to the user's preferences.
    The top 10 results are shown.
  }
  \label{table:rank:paris}
\end{table}



As we have seen in this section, adaptive recommenders can provide personalized search
in multiple ways. 
Because of this, hypotheses H3 is confirmed for these use cases with our dataset.
By varying the IR weight we can create quite a range of systems:
On the one hand, an IR weight of 0 will let the recommenders do all the ranking.
On the other hand, by increasing the IR weight, the recommenders will carefully
adapt parts of the IR results list by moving some of the items.

\begin{table}[t]
  \centering 
  \begin{minipage}{0.49\textwidth}
    \centering 

  \begin{tabular*}{\textwidth}{ r l l }
    \toprule
    \emph{\#} & \emph{score} & \emph{title}\\
    \midrule
    1 &  2.7742  &  Fallen (1998)      \\
    2 &  2.7742  &  Sphere (1998)      \\
    3 &  2.7742  &  Phantoms (1998)    \\
    4 &  2.7742  &  Vermin (1998)      \\
    5 &  2.7742  &  Twilight (1998)    \\
    6 &  2.7742  &  Firestorm (1998)   \\
    7 &  2.7742  &  Palmetto (1998)    \\
    8 &  2.7742  &  The Mighty (1998)  \\
    9 &  2.7742  &  Senseless (1998)   \\
    10&  2.7742  &  Everest (1998)     \\
    \bottomrule
  \end{tabular*}
\end{minipage} 
\hfill 
\begin{minipage}{0.49\textwidth}
  \begin{tabular*}{\textwidth}{ r l l l }
    \toprule
    \emph{\#} & \emph{rating} & \emph{title}\\
    \midrule
    1 &  3.8694  &  Apt Pupil (1998)            \\
    2 &  3.4805  &  The Wedding Singer (1998)   \\
    3 &  3.1314  &  Fallen (1998)               \\
    4 &  3.1225  &  Tainted (1998)              \\
    5 &  2.9442  &  Blues Brothers 2000 (1998)  \\
    6 &  2.9046  &  Sphere (1998)               \\
    7 &  2.8842  &  Desperate Measures (1998)   \\
    8 &  2.8798  &  Firestorm (1998)            \\
    9 &  2.8633  &  Vermin (1998)               \\
    10&  2.8511  &  The Prophecy II (1998)      \\
    \bottomrule
  \end{tabular*}
  \end{minipage} 
  \vspace{1em}
  \caption[Ranking Many Results]{
    Ranking many results: 
    In this example, the user search for the query "1998", to get movies from that year.
    The top 10 of these are shown in the left table. As this query matches a lot of 
    movies, the IR method returns a large number of results. By setting the IR weight to $0$,
    and letting the stacked user models do the ranking, the top 10 results change completely,
    while still being good matches for the current query.
  }
  \label{table:rank:year}
\end{table}


We have not considered which IR weight or other parameters would result in the best performing
personalized search system.
However, this is completely dependent on the type of system and types of queries.
By varying the IR weight, a number of different systems that work for different use cases
can be constructed. For systems with simple one-word queries, setting the weight
to $0$ leaves ranking to the recommenders.
For systems with more complex queries, an IR weight of $1.0$ 
orders items both by IR score and predicted rating.
Naturally, this weight is a defining characteristic of any 
personalized search based on adaptive recommenders.

The performance of personalized search is hard to judge without
extensive query logs with click-through information.
While we had no access to such data, 
we have been able to show that adaptive recommenders
can be used to provide personalized search.
This positive result for Experiment 3 confirms hypothesis H3,
at least for this dataset, this IR system and our chosen recommender algorithms.

The next chapter will discuss the implications and limitation of our results.

