This chapter presents our approach to user modeling:
Blending multiple predictors on a per-user basis.
We will present the reasoning behind or hypotheses,
the rationale of our models,
and explain one possible solution, in detail.
Chapter \ref{chap:results} will experiment with this solution,
before we discuss and draw conclusions in the final chapter.


\section{Hypotheses}

When creating a user modeling system, the researcher(s) find some metric they think can be used
to represent users. In other words,
\emph{user modeling methods are dependent on the subjective assumptions of their creators}.
Each modeling method looks at some aspect of the data, e.g. social connections, similar users,
or a feature of the items (see Chapter \ref{chap:theory}). 
The decision of importance is made by whoever created the modeling method, not by the users themselves.

Aggregate modeling methods face the same problem of misplaced subjectivity: 
Aggregation is done on a generalized, global level,
where each user is expected to place the same importance on each modeling method.
What if this decision could be left up to each user?

Consider the following relevance judgement methods:

\begin{itemize*}
  \item PageRank \citep{Bender2005} assumes that the relevance of a web page is 
  represented by its authority, as computed from inbound links from other sites.
  \item When providing personalized search results, one ranking signal may be 
  the social connections of the current user. Items deemed relevant by the user's 
  peers will then recieve a boosted ranking (e.g. \cite{Carmel2009}).
  \item When determining the relevance of an email, one predictor might be based
  on how often the sender sends emails to the current user.
  \item When recommending movies, one predictor may be based on the ratings
  of users with similar profile details. Another predictor might be 
  dependent on some feature, e.g. production year of well liked movies.
  \item Recommendations based on the Pearson Coefficient \cite[p11]{Segaran2007}
  assumes that the statistical correlation between user ratings defines their 
  similarity.
\end{itemize*}

Are these metrics subjective? While one user might appreciate a social
influence in their search results, another user might not.
While one user might find frequency of communication maps well to relevance,
another might not. 
One user might feel the similarity of movie titles are a good predictor,
while another might be more influenced by production year.
The exact differences are not important --- what is important is that they exist.

As the number of items, users and data sources grow, the problem becomes more apparent.
Creating one or a collection of modeling methods that accuratly distills the general
taste or judgement of all these elements should be regarded with skepticism.
Likewise, as the number of modeling methods increase, finding an optimal combination
becomes a gargantuan problem, requiring deep and often unattainably correct domain knowledge.

We propose a more individual approach: \emph{personalized predictor aggregation}.
By introducing an extra level of abstraction, we can in essence leave the method of aggregation
up to each user. By selecting how the different modeling methods will be applied based
on their respective individual performance, we change the problem faced by the system:
The question is no longer which modeling methods we should choose or how they should be combined,
but rather finding and adding \emph{any} modeling method a user might find helpful.

If each method is \emph{only used} based on how well it performs for each user,
any possibly applicable user modeling method suddenly becomes a worthy addition.
We believe this is a much simpler problem: 
instead of trying to generalize individuality,
it should be embraced, by allowing users to implicitly and automatically select which methods they prefer.

In light of these assumptions, we will try to answer the following hypotheses:

$H_{0}$: The accuracy of user-item relevance predictions can be improved
by blending multiple modeling methods on a per-user basis.

$H_{1}$: The result set from an information retrieval query
can be personalized by blending multiple modeling methods on a per-user basis.



\section{Modeling}

A system for aggregate user modeling (AUM) can be described as a quintuple:

\begin{eqnarray*}
  \mathrm{AUM} &=& (Items, Users, Ratings, Framework, Methods, Aggregation)\\
               &=& (I,U,R,F,M,A).
\end{eqnarray*}

We have a set of $Items$ and a set of $Users$.
There is also a set of $Ratings$: each user $u \in U$ can \emph{rate} an item $i \in I$.
We use the term "rating" loosely --- other applicable and equivalent terms include \emph{relevance}, \emph{utility},
\emph{connection strength} or \emph{ranking}. In other words, this is a measure of what a user thinks of an item
in the current domain language. However, since "rating" will match the case study we present later in this chapter,
that is what we shall use.

As we are dealing with multiple approaches to user modeling, we have a set of $Methods$ that each create their own
user models. Each model $m \in M$ are used to compute predictions, i.e. estimations of unknown ratings.
As demonstrated in Chapter \ref{chap:theory}, there are many different forms of user modeling,
that each consider differents aspects of the available data: the users, items and ratings, as well as 
other sources such as intra-user connections in social networks or intra-item connections in information retrieval systems.

The $Aggregation$ part of this quintuple refers to how the predictions from the different methods are combined
into one blended prediction. 
This paper presents a per-user, individual approach to this aggregation. However, there are many, simpler ways of computing the
combined prediction. For example, a weighted average could be constructed as simply as 

\begin{equation*}
  \hat{r}_{ui} = \sum_{m \in M} w_m \cdot p_m(u,i) 
  \quad \text{where}
  \quad 0 \leq w \leq 1, 
  \quad \sum_{i} w_i = 1.
\end{equation*}

Here, $\hat{r}_{ui}$ is the predicted rating from user $u$ of item $i$.
$w_m$ is the weight of method $m$ and $p_m(u,i)$ is the predicted rating from method $m$.
Of course, the weights of each method have to be estimated in advance.
By creating a training and a testing set from already known ratings, we can estimate the weights
that minimize the error over the testing set. 
This can for example be done through \emph{regression} or the \emph{least squares method}.
(TODO: Refs)




\clearpage

\begin{equation*}
  \mathrm{AM} = (Items, Users, Framework, Methods, Aggregation)
\end{equation*}

\section{Prediction}

User-weighted average

\begin{equation*}
  \hat{r}_{ui} = \sum_{m \in M} w_{um} \cdot p_m(u,i)
\end{equation*}


Higher-order

\begin{equation*}
  \hat{r}_{ui} = f_{u}[ P_{M}(u,i) ]
\end{equation*}





\section{Implementation}      



