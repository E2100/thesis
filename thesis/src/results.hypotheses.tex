\section{Three Hypotheses}
\label{sec:hypotheses}

To evaluate whether or not \emph{stacked recommenders}
can mitigate the latent subjectivity problem,
we need a set of hypotheses.
To solve the subjectivity problem we need our modeling method
to adaptively aggregate a set of predictions based on the current user and item.
In other words, by automatically adapting how a set of disjoint recommenders
are combined, based on each user and item, we should be able to achieve a
result that is better than each of the stand-alone recommenders.
This adaptive method should also outperform other, generalized aggregation approaches.

This chapter will consider three hypotheses (H1-H3) 
to establish whether adaptive aggregation through stacked recommenders is a viable technique.
H1 and H2 will consider the approach in regard to prediction aggregation
in a recommendation scenario. H3 will consider using this approach for
rank aggregation in an information retrieval scenario.
Let us start with prediction aggregation:

{
  \itshape
  $\mathbf{H1}$: The accuracy of relevance predictions can be improved
  by blending multiple modeling methods on a per-user and per-item basis.
}

It stands to reason that if a recommender system is indeed impaired
by the subjective selection of modeling methods,
an adaptive combination of these methods should outperform each of the individual approaches.
Second:

{
  \itshape
  $\mathbf{H2}$: An adaptive aggregation method can outperform global and generalized 
  blending methods.
}

If our assumption that model aggregation inherits the subjective nature of its chosen parts,
an adaptive aggregation without such misplaced subjectivity should outperform a
generalized combination.
Third:

{
  \itshape
  $\mathbf{H3}$: The result set from an information retrieval query
  can be personalized by stacking recommenders, where the retrieval scores are considered 
  standard input signals.
}

As described in Section \ref{subsec:signals},
modern search engines combines multiple ranking functions called signals into a final results list.
Each signal, be it an IR score, a metric like PageRank, or the result of a user modeling method,
contributes to the final item ranking.
We shall use H3 to see whether or not stacked recommenders can be used for this type of rank aggregation,
where a set of recommender systems constitute a set of input signals.

By answering our three hypotheses, it should become clear whether or not our approach
is a valid method of predicting the relevance of an item to a user.
These hypothesess will be tested in the next chapter, but first, let us build
the model we shall test.

