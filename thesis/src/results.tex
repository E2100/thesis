\label{chap:results}

This chapter will perfom experiments to find out whether or not
\emph{stacked user modeling} is a viable technique.
We will first explain which evaluation metrics will be used,
and the datasets we will use to test the technique.
Then, we will test each of hypotheses given in Section \ref{sec:hypotheses}.
The next chapter will discuss the implications of these results.


\section{Evaluation Metrics}

To evaluate how our model performs, we need a measure
for computing the total error across a large number of predictions.
The canonical measure for estimating the error of a 
predictions from a recommender system
is the \emph{Root mean squared error} (RMSE) measure
(e.g. \citet[p17]{Herlocker2004}, \citet[p13]{Adomavicius2005} and \citet[p6]{Bell2007}).
We shall use this measure to estimate the performance
of our adaptive prediction aggregation algorithms.
The RMSE of a set of estimations $\hat{R}$, 
compared to a set of known ratings $R$, is defined as

\begin{equation*}
  RMSE(\hat{R},R) = \sqrt{E((\hat{R} - R)^2)}
  = \sqrt{\frac{
      \sum_{i=1}^{n} (\hat{R}_i - R_i)^2
    }{
      n
    }},
\end{equation*}

where $n$ is the total number of predictions.
The RMSE combines a set of errors into one single combined error.
A beneficial feature of the RMSE is that the resulting error 
will be on the same scale as the estimations. In other words,
if we are predicting values on the scale $1-5$, the computed error
will be on this scale as well. In this case, an error of $1$
would then say that we are on average $1$ point away from the true 
ratings on our $1-5$ scale.

While the RMSE works well for evaluating scalar predictions,
we need another measure for considering a predicted sorting 
from rank estimation methods.
Here, we are not interested in the predicted scores,
but rather in which position each item appears in a sorted list of results.
This is for instance needed when measuring the performance of a
personalized search engine.
Because of this, we are interested in examining how 
personalization with stacked user modeling affects the rankings
from an IR method.


\section{Datasets}

To test our model, we need an applicable dataset with users, items and ratings.
We chose the MovieLens dataset\footnote{
See http://www.grouplens.org/node/73 --- accessed 10.05.2011} of movie ratings.
This dataset is often used to test the performance of recommender systems,
for example in \citet[p9]{Alshamri2008}, \citet[p4]{Lemire2005}, \citet[p1]{Adomavicius2005}
and \citet[p2]{Herlocker2004}.
The dataset consits of a set of users, a set of movies, and a set of movie ratings
on the scale $1$ through $5$, and is available in two sizes:

\begin{itemize*}
  \item A set of 100,000 ratings from 943 users on 1,682 movies.
  \item A set of 100,000,209 ratings from 6,040 users of 3,900 movies.
\end{itemize*}

Each dataset also comes with meta-data on each user, such as
gender, age and occupation. There is also meta-data on each movie,
such as its title, release date and genre. 
For prediction aggregation, we are only interested in the ratings matrix
extracted from this dataset, while the titles of each movie will be used
to experiment with personalized search.



\section{Prediction Aggregation}

\begin{table}
  \begin{tabular*}{\textwidth}{ l l }
    \toprule
    \emph{method} & \emph{description} \\
    \midrule
    svd1          & SVD recommender, ALSWR factorizer and 10 factorizer features. \\
    svd2          & SVD recommender, ALSWR factorizer and 20 factorizer features. \\
    svd3          & SVD recommender, EM factorizer and 10 factorizer features. \\
    svd4          & SVD recommender, EM factorizer and 20 factorizer features. \\
    slope\_one    & Standard Slope One rating difference recommender. \\
    item\_avg     & Baseline recommender based on the average ratings of items. \\
    average       & Baseline recommender based on the average ratings of items and users. \\
    cosine   	    & Weighted ratings based on cosine similarity between items. \\
    knn       	  & Collaborative predictions weighted by user similarity through the PCC. \\
    \midrule
    median    	  & Aggregation: The median rating from the above methods. \\
    avg       	  & Aggregation: The average rating from the above methods. \\
    stacked       & Adaptive aggregation: Secondary, SVD-based accuracy predictions. \\
    \bottomrule
  \end{tabular*}
  \caption[Stacked Modeling Methods]{
    Stacked modeling methods: A short description of each of the recommender methods
    used in our experiment. See Section \ref{sec:recommender} for more information.}
  \label{table:results:methods}
\end{table}

\begin{table}
  \begin{tabular*}{\textwidth}{ l p{3cm} p{1.5cm} p{1.5cm} p{1.5cm} p{1.5cm} p{1.5cm} }
    \toprule
      ~ & \emph{method} & 
      $d_1$ & $d_2$ & $d_3$ & $d_4$ & $d_5$ \\ 
    \midrule
    S & svd1          & 1.2389	  & 1.1260	  & 1.1327	  & 1.1045	  & 1.1184	 \\
    S & svd2          & 1.2630	  & 1.1416    & 1.1260	  & 1.1458	  & 1.1260	 \\
    S & svd3          & 1.0061	  & 0.9825	  & 0.9830	  & 0.9815	  & 0.9797	 \\
    S & svd4          & 1.0040	  & 0.9830	  & 0.9849	  & 0.9850	  & 0.9798	 \\
    S & slope\_one    & 1.1919	  & 1.0540	  & 1.0476	  & 1.0454	  & 1.0393   \\
    S & item\_avg     & 1.0713	  & 0.9692	  & 0.9662	  & 0.9683	  & 0.9725	 \\
    S & average       & 1.0698	  & 0.9557	  & 0.9527	  & 0.9415	  & 0.9492	 \\
    S & cosine   	    & 1.1101	  & 0.9463	  & 0.9412	  & 0.9413	  & 0.9382	 \\
    S & knn       	  & 1.4850	  & 1.1435	  & 1.1872    & 1.2156	  & 1.2022	 \\
    \midrule                                                                    
    A & median    	  & 0.9869	  & 0.8886	  & 0.8857    & 0.8857	  & 0.8855	 \\
    A & avg       	  & 0.9900	  & 0.8536	  & 0.8525	  & 0.8525	  & 0.8519	 \\
    A & stacked       & \textbf{0.9324}	  & \textbf{0.8015}	  & \textbf{0.7993}  & \textbf{0.8238} & \textbf{0.8192} \\
    \bottomrule
  \end{tabular*}
  \caption[Results]{RMSE, H1, H2}
  \label{table:results:h2}
\end{table}

\begin{table}
  \begin{tabular*}{\textwidth}{ l p{3cm} p{2cm} p{2cm} p{2cm} p{2cm} }
    \toprule
      ~ & \emph{method} & 
      \emph{min} & \emph{max} & \emph{mean} & $\sigma$\\
    \midrule
    S & svd1          & 1.1045	& 1.2389	& 1.1441	& 0.2197 \\
    S & svd2          & 1.1260	& 1.2630	& 1.1605	& 0.2277 \\
    S & svd3          & 0.9797	& 1.0061	& 0.9865	& 0.0991 \\
    S & svd4          & 0.9798	& 1.0040	& 0.9873	& \textbf{0.0924} \\
    S & slope\_one    & 1.0393	& 1.1919	& 1.0756	& 0.2415 \\
    S & item\_average & 0.9662	& 1.0713	& 0.9895	& 0.2023 \\
    S & average       & 0.9415	& 1.0698	& 0.9738	& 0.2196 \\
    S & cosine   	    & 0.9382	& 1.1101	& 0.9754	& 0.2595 \\
    S & knn       	  & 1.1435	& 1.4850	& 1.2467	& 0.3487 \\
    \midrule            
    A & median    	  & 0.8855	& 0.9865	& 0.9065	& 0.2005 \\
    A & average    	  & 0.8519	& 0.9900	& 0.8801	& 0.2344 \\
    A & stacked       & \textbf{0.7993}	& \textbf{0.9324}	& \textbf{0.8352}	& 0.2225 \\
    \bottomrule
  \end{tabular*}
  \caption[Results]{RMSE stats, H1, H2}
  \label{table:results:h2:sum}
\end{table}

H1

H2


\section{Rank Aggregation}

Test H3


%\section{Results}
%Summarize results


