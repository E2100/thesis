\section{Hypotheses}

So far we have taken a look at approaches to recommendations and personalized search:
Traditionally, a single method that considers some pattern in the data is used.
Modern techniques blend different methods to achieve a combined accuracy 
that outranks the performance of each individual method.

However, based on the notion of latent subjectivity in method selection and aggregation,
we desire an even more personalized technique. 
To this end, this paper will test three hypotheses (H1-H3), in an effort to establish
whether per-user aggregation is a viable technique. First:

{
  \itshape
  $\mathbf{H1}$: The accuracy of user-item relevance predictions can be improved
  by blending multiple modeling methods on a per-user basis:
  $\forall m \in M: \mathrm{error}(am_u) < \mathrm{error}(m)$.
}

It stands to reason that if a recommender system is indeed impaired
by the subjective selections of modeling methods,
a per-user blend of these methods should outperform each of the individual approaches.
Second:

{
  \itshape
  $\mathbf{H2}$: A per-user aggregation method can outperform global and generalized 
  blending methods:
  $\forall a \in A: \mathrm{errorr}(am_u) < \mathrm{error}(a)$.
}

If our assumption that model aggregation inherits the subjective nature of its chosen parts,
a per-user aggregation without such misplaced subjectivity should outperform a
generalized blending.
Third:

{
  \itshape
  $\mathbf{H3}$: The result set from an information retrieval query
  can be personalized by blending multiple modeling methods on a per-user basis.
}

As described in Section \ref{subsec:signals},
every measure that influence the final ranking of results from a search engine
can be seen as individual signals. Each signal, be it an IR score,
something like PageRank or the result of a user modeling method,
contributes to the final ranking.
As the aggregation of these signals are the same problem faced
when aggregating recommender systems,
a per-user approach to blending signals should be able to
produce a set of personalized search results.
In other words, techniques from information retrieval and user modeling
should be combined on a per-user basis to provide individually adapted results.

