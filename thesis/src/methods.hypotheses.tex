\section{Hypotheses}

To minimize any latent subjectivity, we would like a method
for prediction aggregation that works on a per-user and per-item level (\emph{per-element}).
By automatically adapting each modeling method based on which user and which item is currently being
considered, we should be able to produce an improved result.
This paper will consider three hypotheses (H1-H3) 
to establish whether adaptive aggregation through stacked user modeling is a viable technique:

{
  \itshape
  $\mathbf{H1}$: The accuracy of relevance predictions can be improved
  by blending multiple modeling methods on a per-user and per-item basis.
}

It stands to reason that if a recommender system is indeed impaired
by the subjective selections of modeling methods,
an adaptive blend of these methods should outperform each of the individual approaches.
Second:

{
  \itshape
  $\mathbf{H2}$: An adaptive aggregation method can outperform global and generalized 
  blending methods.
}

If our assumption that model aggregation inherits the subjective nature of its chosen parts,
a per-user aggregation without such misplaced subjectivity should outperform a
generalized blending.
Third:

{
  \itshape
  $\mathbf{H3}$: The result set from an information retrieval query
  can be personalized by stacked user modeling, where the retrieval scores are considered 
  standard input signals.
}

As described in Section \ref{subsec:signals},
every measure that influence the final ranking of results from a search engine
can be seen as individual signals. Each signal, be it an IR score,
something like PageRank or the result of a user modeling method,
contributes to the final ranking.

This fits well into our thinking that multiple modeling methods
will leverage multiple patterns in the data.
The information retrieval ranking function
becomes another metric, just like the user modeling methods,
allowing us to do IR and personalization with the same algorithm.

\clearpage

