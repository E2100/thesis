\section{Three Hypotheses}
\label{sec:hypotheses}

Our research goal is to develop the \emph{adaptive recommenders} technique, and determine if this is a viable approach.
To solve the subjectivity problem we need our modeling method
to adaptively aggregate a set of predictions based on the current user and item.
In other words, by automatically adapting how a set of disjoint recommenders
are combined, based on each user and item, we should be able to achieve a
result that is better than each of the stand-alone recommenders.
This adaptive method should also outperform other, generalized aggregation approaches.

In order to achieve our goal, this thesis will consider three hypotheses (H1-H3).
H1 and H2 will consider the approach in regard to prediction aggregation
in a recommendation scenario. H3 will consider using this approach for
rank aggregation in an information retrieval scenario.
Let us start with prediction aggregation:

\begin{blockquote}
  H1: The accuracy of relevance predictions can be improved
  through adaptive recommender aggregation.
\end{blockquote}
%
It stands to reason that if a recommender system is indeed impaired
by the subjective selection of modeling methods,
an adaptive combination of these methods, based on each individual user and item, 
should outperform each of the individual approaches.
Second:

\begin{blockquote}
  H2: Adaptive aggregation can achieve higher accuracy than global and generalized aggregation methods.
\end{blockquote}
%
If our assumption that model aggregation inherits the subjective nature of its chosen parts,
an adaptive aggregation without such misplaced subjectivity should outperform a
generalized combination.
Third:

\begin{blockquote}
  H3: The result set from an information retrieval query
  can be adaptively personalized by layering recommenders.
\end{blockquote}
%
As described in Section \ref{subsec:signals},
modern search engines combines multiple ranking functions called signals into a final results list.
We shall use H3 to see whether or not adaptive recommenders can be used for this type of rank aggregation,
where a set of recommender systems constitute a set of input signals.

By answering these three hypotheses, it chould become clear whether or not
our \emph{adaptive recommenders} is a viable technique for improved relevance predictions
between users and items.

