\section{Prediction Aggregation}

Adaptive prediction aggregation means combining the results
from multiple scalar predictors based on the current user and item
(see Section \ref{sec:theory:predictionagg}).
As mentioned, we have two levels of predictors:
The first level is a set of traditional recommender systems
that produce estimations of unknown ratings between users and items.
The second level is another set of recommender systems 
that predict how accurate each of the first level recommenders will be.

In this section, we shall explain how such a system may be created.
Most importantly, there are two distinct phases to adaptive recommenders:

\begin{enumerate*}
  \item The modeling phase creates the user models for both levels.
  \item The prediction phase uses the created models to estimate ratings.
\end{enumerate*}

We shall first explain the modeling phase, then the prediction phase,
when dealing with prediction aggregation.
The next section will explain a similar situation where
we wish to do \emph{adaptive rank aggregation}: 
combining ordered lists of results, depending on the current user and item.


\subsection{Modeling Phase}

Listing \ref{code:training} gives the basic algorithm for training
our models. The input to this method is the standard ratings matrix,
and a set of untrained modeling methods (in this case,
untrained recommender systems).

\begin{algorithm}
  \begin{algorithmic}[1]
  \REQUIRE ratings: The ratings matrix
  \REQUIRE methods: The set of modeling methods
  \ENSURE
    \STATE $rating\_models \gets \emptyset$
    \STATE $error\_models \gets \emptyset$
    \FORALL{$m \in methods$}
      \STATE $sample \gets \mathrm{BootstrapSample}(ratings)$
      \STATE $rating\_models_m \gets \mathrm{TrainModel}(m, sample)$
      \STATE $error\_models_m  \gets \mathrm{TrainErrorModel}(rating\_models_m, ratings)$
    \ENDFOR 
  \RETURN $(rating\_models, error\_models)$
  \end{algorithmic}
  \caption[Adaptive Prediction Aggregation Modeling]{Adaptive Prediction Aggregation Modeling
  }
  \label{code:training}
\end{algorithm}

An important question is how we should split the ratings data.
In this scenario, we need to split the data for a number of purposes.
The following sets must be created during training:

\begin{enumerate*}
  \item Training sets to create the standard recommenders.
  \item Training sets to create the error estimation models.
  \item A testing set to test our final system.
\end{enumerate*}

Constructing these subsets of the available data is a common task in ensemble learning
\cite[p7]{Polikar2006}.
As seen in Listing \ref{code:training}, we use an approach called 
\emph{bootstrap aggregation}, also known as \emph{bagging}
(introduced by \cite{Breiman1996}).
Originally, bagging is used by ensemble learning classification methods, where multiple classifiers are 
trained by uniformly sampling a subset of the available training data. 
Each model is then trained on one of these subsets, and the models are aggregated by averaging their individual predictions.

Formally, given a training set $D$ with $n$ items, bagging creates $m$ new training sets of size $n' \leq n$ by sampling
items from $D$ uniformly and with replacement. 
In statistics, these types of samples are called \emph{bootstrap samples}.
If $n'$ is comparable in size to $n$, there will be some items
that are repeated in the new training sets.

Bagging suits our needs perfectly, for a few reasons: First, the method helps create disjoint predictors, 
since each predictor is only trained (or specialized for) a subset of the available data.
Second, it allows us to easily train the underlying modeling methods without any complex partitioning of the data.
Our partitioning strategy is now clear:

\begin{enumerate*}
  \item Split the entire dataset into a training and testing set.
  \item Train modeling methods through bootstrap aggregation of the training set.
  \item Train error models from the complete training set.
  \item Test the resulting system with the initial testing set.
\end{enumerate*}

Each modeling method is trained in ways specific to their implementation. 
Model-based approaches create pre-built structures and provide offline training,
while heuristic methods simply store the data for future computation.
Either way, it is up to each modeling method what it does with the supplied training data.
The result of this algorithm is a set of trained rating models and error models.

\begin{algorithm}
  \begin{algorithmic}[1]
  \REQUIRE ratings: The ratings matrix
  \REQUIRE rating\_model: A standard user model
  \ENSURE
    \STATE $errors \gets [[]]$
    \FORALL{$user,item,rating \in ratings$}
        \STATE $errors_{user,item} \gets | ratings_{user,item} - \mathrm{Predict}(rating\_model, user, item) |$
    \ENDFOR 
    \STATE $error\_method \gets \mathrm{NewModelingMethod}(SVD)$
    \STATE $error\_model  \gets \mathrm{TrainModel}(error\_method, errors)$
  \RETURN $error\_model$
  \end{algorithmic}
  \caption[Prediction Error Modeling]{Prediction Error Modeling}
  \label{code:trainerrormodel}
\end{algorithm}

Listing \ref{code:trainerrormodel} shows an algorithm for training the error models.
The input is the entire ratings matrix, and a trained recommender model
that this error model should represent.
We first create the aforementioned error matrix by estimating
predictions for each known combination in the ratings data.
The $\mathrm{NewModelingMethod}$ call simply creates a new, untrained
recommender model of some pre-specified $type$
(in this case, a new SVD-based model, but any recommender method will do).
A new model is then trained based on the created error matrix,
and returned as our new $error\_model$.

When the computations of the algorithm in Listing \ref{code:training} is complete,
we have a set of trained recommender systems, and a set of trained error models.
Each recommender model has a corresponding error model,
forming two layers, that we shall use when performing predictions.


\subsection{Prediction Phase}

In the prediction phase of adaptive prediction aggregation,
we wish to use our layers of trained models to produce adaptive
combinations of multiple predictions and accuracy estimations.
Listing \ref{code:prediction} gives the basic algorithm.

\begin{algorithm}
  \begin{algorithmic}[1]
  \REQUIRE user, item: A user and an item
  \REQUIRE rating\_models: The set of trained modeling methods 
  \REQUIRE error\_models: The set of trained error models
  \ENSURE
    \STATE $ratings \gets \emptyset$
    \STATE $errors  \gets \emptyset$
    \FORALL{$m \in rating\_models$}
      \STATE $ratings_{m} \gets \mathrm{Predict}(rating\_models_m, user, item)$
      \STATE $errors_{m}  \gets \mathrm{Predict}(error\_models_m, user, item)$
    \ENDFOR 
    \STATE $errors \gets \mathrm{Normalize}(errors)$
    \STATE $prediction \gets 0$
    \FORALL{$m \in rating\_models$}
      \STATE $weight_m \gets 1 - error_m$
      \STATE $prediction \gets prediction + weight_m \times ratings_m$
    \ENDFOR
 
  \RETURN $prediction$
  \end{algorithmic}
  \caption[Adaptive Prediction Aggregation]{Adaptive Prediction Aggregation
  }
  \label{code:prediction}
\end{algorithm}

The first input is the user and item for which we wish to predict a rating.
We assume that this rating is unknown --- predicting ratings for known combinations
would mean recommending items the user has already seen and considered
(however, if we are dealing with a task such as personalized search,
these known ratings are of course important, as we shall see in the next section).

The other inputs are the trained rating models, and the corresponding error models.
The algorithm begins by creating empty sets for predicted ratings and errors.
Next, each modeling method is used to predict ratings, and their error models to predict errors.
Note that each step in the first for-loop is independent of each other, and both steps
inside the for loop is also independent. This is then an algorithm well suited for
parallelization.
In a MapReduce framework, this for loop would be run as a $\mathrm{map}$ operation,
where the input user and item is mapped over the sets of modeling methods
(see Appendix \ref{appendix:implementation} for implementation details).

After the predictions have been collected, the errors are normalized,
i.e. converted to the range $[0,1]$ and changed to sum to $1$.
This is vital before last stage of the prediction algorithm,
which weighs each prediction from the different rating models.
This step corresponds to the previously explained $\mathrm{reduce}$ operation,
that combines multiple scores into one final result.
The weight of each method is computed as $1 - error$, where $error$ 
is the normalized error for this method, for the current user and item.
Each rating prediction is then weighted, and combined to form the final,
adaptively aggregated prediction.

There is an important performance different between the modeling and prediction phases:
While the modeling phase is the most computationally expensive,
it can be performed independently of making predictions.
As the prediction phase is when the user has to wait for the system,
this is of course where performance is most important.
Naturally, as users rate more items and new items arrive,
the models have to be recreated based on this new reality.
However, as the modeling phase is an offline operation,
the training can be performed in the background, while new
and speedy predictions are always at the users hands.



