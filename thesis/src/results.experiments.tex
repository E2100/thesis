\section{Three Experiments}

Table \ref{table:experiments} shows the experiments performed in this chapter.
Experiments 1 \emph{\&} 2 will test hypotheses H1 \emph{\&} H2, through prediction aggregation.
Experiment 3 will test hypothesis H3 through rank aggregation.
The first two experiments will be quantitative measurements of performance during prediction aggregation
The last experiment will be a qualitative exploration
of how to do personalized search with stacked recommenders.
In particular, we will look at how different 
prioritizations of the IR model scores influence the
final sorted list.

\vspace{1em}
\begin{table}[h]
  \begin{tabular*}{\textwidth}{ l l l l l l l }
    \toprule
      ~ & 
      \emph{mission} &
      \emph{hypotheses} &
      \emph{dataset} &
      \emph{users} &
      \emph{items} &
      \emph{ratings} \\
    \midrule
    
    Experiment 1 &
    pred.agg. &
    H1, H2 &
    MovieLens &
    943 &
    1,682 &
    100,000 \\
    
    Experiment 2 &
    pred.agg. &
    H1, H2 &
    Jester &
    24,983 &
    100 &
    1,832,275 \\

    Experiment 3 &
    rank.agg. &
    H3 &
    MovieLens &
    943 &
    1,682 &
    100,000 \\

    
    \bottomrule 
  \end{tabular*}
  \caption[List of Experiments]{List of Experiments performed in this chapter.}
  \label{table:experiments}
\end{table}


As seen in Table \ref{table:experiments}, 
we will use two distinct datasets to perform experiments.
Each dataset have different numbers of items, users and ratings.
This is a desirable property, as datasets with different
characteristics will help us achieve more reliable results.

(1) First is the MovieLens dataset\footnote{
See http://www.grouplens.org/node/73 --- accessed 10.05.2011}.
This dataset is often used to test the performance of recommender systems,
as described in 
\citet[p9]{Alshamri2008}, \citet[p4]{Lemire2005}, \citet[p1]{Adomavicius2005} and \citet[p2]{Herlocker2004}.
The dataset consits of a set of users, a set of movies, and a set of movie ratings
on the scale $1$ through $5$, and is available in multiple sizes.
We chose the set with 100,000 ratings from 943 users on 1,682 movies.

The MovieLens dataset also comes with meta-data on each user, such as
gender, age and occupation. There is also meta-data on each movie,
such as its title, release date and genre. 
For Experiments 1, we are only interested in the ratings matrix
extracted from this dataset.
The titles of each movie will be used 
in Experiment 3 to achieve personalized search.

(2) Our second set of ratings comes from the Jester dataset\footnote{
See \url{eigentaste.berkeley.edu/dataset/} ---
accessed 22/05/2011}.
This is a set of items rated by users on a continous scale.
As with MovieLens, this dataset is also commonly used
to test recommender systems (as described in
\cite{Goldberg2001}, \citet[p14]{Herlocker2004}, \citet[p5]{Adomavicius2005} and \citet[p30]{Ahn2004}).
This dataset is considerably larger than our first set,
consisting of 1,832,275 ratings from 24,983 users of 100 items.
Notably, the number of items is quite smaller than in the other dataset.

The scale is also different from the MovieLens dataset:
while each movie is rated on a discrete scale from $1$ through $5$,
the items in Jester are rated on a continous scale from $-10$ to $10$.
However, in order to compare the measurements on both datasets,
the ratings in Jester were converted to be on the scale $1-5$.
Still, the difference between ordinal and continous ratings remains,
and will give us another differing quality to verify our results.

We will use the Jester dataset to test hypotheses H1 \emph{\&} H2.
In other words, two distinct datasets will be used for these hypotheses,
in an effort to further verify our results.

In another effort to achieve reliable evaluation results, 
both dataset was split into multiple disjoint subsets, so that we can do cross-validation.
This entails running the same experiments across all the subsets,
and averaging the results.
Each dataset is split into five disjoint subsets,
which are again split into training and testing sets:

\begin{eqsp}
  D = \{ d_1 = \{base_1, test_1\}, d_2 = \{base_2, test_2\}, ..., d_5 = \{base_5, test_5\} \}
\end{eqsp}
%
Each $base_x$ and $test_x$ are disjoint 80\% / 20\% splits of the data in each subset.
We shall perform five-fold cross-validation across all these sets in our experiments.
This way we can be more certain that our results are reliable,
and not because of local effect in parts of the data.

As previously explained, each $base$ set is further split using bootstrap aggregation,
into random subsets for training each stanard recommender model.
The entire base set is then used to train the aggregation models.
Each corresponding $test$ set is then used to evaluate the performance
of each basic model, and each of the aggregators.

Before diving into each experiment, let us take 
a closer look at the different types of recommender systems
they will use.

