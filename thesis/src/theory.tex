This chapter introduces the basic theories of 
user modeling, recommender systems, personalized search and user model aggregation.

\section{User Modeling}

User Modeling (UM) is all about adapting an application to its users.
This has the potential to solve three problems that many applications face:

\begin{enumerate}

\item \emph{Information overload} conveys the act of receiving \emph{too much information} \cite[p13]{Bjorkoy2010d}. 
The problem is apparent in situations where decisional accuracy turns from improving with more information, to being hindered by too much irrelevant data \cite{Eppler2004}. 
The overload is often likened to a \emph{paradox of choice}, as there may be no problem acquiring the relevant information, but rather identifying this information once acquired. As put by \cite{Edmunds2000}: "The paradox --- a surfeit of information and a paucity of useful information."
While normal cases of such overload typically result in feelings of being overwhelmed and out of control, \cite{Bawden2008} points to studies linking extreme cases to various psychological conditions related to stressful situations, lost attention span, increased distractibility and general impatience.
\footnote{Information overload is a widespread phenomenon, with as many definitions as there are fields experiencing the problem. Examples include \emph{sensory overload}, \emph{cognitive overload} and \emph{information anxiety} \citep{Eppler2004}.}

\item \emph{Content discovery} is closely related to information overload. When the amount of information grows, 
discovering and evaluating new content becomes more difficult. 

\item \emph{Presentational preferences} differ between users. From simple preferences to actual physical contraints. 

\end{enumerate}

\cite{Kirsh2000} argues that "the psychological effort of making hard decisions about \emph{pushed} information is the first cause of cognitive overload." According to \citeauthor{Kirsh2000}, there will never be a fully satisfiable solution to the problem of overabundant information, but that optimal environments can be designed to increase productivity and reduce the level of stress through careful consideration of the user's needs. 

The fields of Artificial Intelligence (AI) and Human-Computer Interaction (HCI) share a common goal solving these problems through user modeling. 
However, as described by \cite{Lieberman2009}, their efforts are seldom combined: while ai researchers often view contributions from hci as trivial cosmetics, the hci camp
tends to view ai as unreliable and unpredictable --- surefire aspects of poor interaction design.

In AI, user modeling refers to precise algorithms and methods that infer knowledge about a user based on past interaction 
(\cite{Pazzani2007, Smyth2007, Alshamri2008, Resnick1994})  
By examining previous actions, predictions can be made of how the user will react to future information. This new knowledge is then embedded in a model of the user, which can predict future actions and reactions. 
For instance, an individual user model may predict how interesting an unseen article will be to a user, based on previous feedback on similar articles or the feedback of similar users.

HCI aims to meet user demands for interaction. 
User modeling plays a crucial role in this task. 
Unlike the formal user modeling methods of AI, user models in HCI are often cognitive approximations, manually developed by researchers to describe different types of users 
(\cite{Fischer2001, Jameson2009, Cato2001})
These models are then utilized by interaction designers to properly design the computer interface based on a models predictions of its user’s preferences.
\cite{Totterdell1990} describes user modeling in interaction design as a collection of deferred parameters: "The designer defers some of the design parameters such that they can be selected or fixed by features of the environment at the time of interaction [...] Conventional systems are special cases of adaptive systems in which the parameters have been pre-set."

In earlier years, the field of user modeling was fragmented by its multidiscipline nature. 
However, as described by \cite{Kobsa2001}, recent research has blurred the lines between the AI and HCI in user modeling.

\begin{eqnarray}
  \mathrm{UM} = (Items, Users, Models, Prediction)
\end{eqnarray}









the modeling problem: model+prediction

the core problem: estimating preferences

getting past 80\%

the efficiency of data


\subsection{Recommender Systems}

\begin{eqnarray}
  \mathrm{RS} = (Items, Users, Ratings, Framework, Method)
\end{eqnarray}

Users, items and ratings

machine learning fundamentals

modeling or heuristics

Taxonomy: model/heur, granularity, temporality, agents

\subsection{Approaches}

Modeling approaches

Heuristic approaches


\section{Personalized Search}

Information retrieval (+ information overload)

An Information Retrielal Model is a quadruple \citep[p23]{Baeza-Yates1999}:

\begin{eqnarray}
  \mathrm{IR} = (Documents, Queries, Framework, ranking(q_i, d_i))
\end{eqnarray}

Common metrics

Personalized metrics

Relation to user modeling


\section{Aggregate Modeling}

\begin{eqnarray}
  \mathrm{AM} = (Items, Users, Framework, Methods, Aggregation)
\end{eqnarray}




Use cases (netflix, ...)

For personalized search (speculation)

Hypotheses

Explain next chapter


