In 1971, Herbert Simon said the following on the topic of information overload: 
"What information consumes is rather obvious: it consumes the attention of its recipients. 
Hence a wealth of information creates a poverty of attention and 
a need to allocate that attention efficiently among the overabundance of 
information sources that might consume it." \cite{Greenberger1971}.


In 2009, Alon Halevy, Peter Norvig, and Fernando Pereira, wrote:
"Perhaps when it comes to natural language processing and related fields, 
we’re doomed to complex theories that will never have the elegance of physics equations. 
But if that’s so, we should stop acting as if our goal is to author extremely elegant theories, 
and instead embrace complexity and make use of the best ally we have: 
the unreasonable effectiveness of data."
\cite{Halevy2009}.

"Predictive accuracy is substantially improved when blending multiple predictors. 
Our experience is that most efforts should be concentrated in deriving substantially 
different approaches, rather than refining a single technique. 
Consequently, our solution is an ensemble of many methods."
--- \cite{Bell2007}



Previously, something else was the problem

Today, biggest problems on the web

Information overload

Content discovery

Search

User modeling

Often generic methods

the modeling problem: model+prediction

the core problem: estimating preferences

getting past 80\%

the efficiency of data

This paper: A more personal approach

Aggregated user modeling methods for 
truly personal predictions.

Hypothesis

Contributions

Outline


