\begin{table}[t]
  \centering 
  \begin{minipage}{0.49\textwidth}
    \centering 

  \begin{tabular*}{\textwidth}{ r l l }
    \toprule
    \emph{\#} & \emph{score} & \emph{title}\\
    \midrule
    1 &  2.7742  &  Fallen (1998)      \\
    2 &  2.7742  &  Sphere (1998)      \\
    3 &  2.7742  &  Phantoms (1998)    \\
    4 &  2.7742  &  Vermin (1998)      \\
    5 &  2.7742  &  Twilight (1998)    \\
    6 &  2.7742  &  Firestorm (1998)   \\
    7 &  2.7742  &  Palmetto (1998)    \\
    8 &  2.7742  &  The Mighty (1998)  \\
    9 &  2.7742  &  Senseless (1998)   \\
    10&  2.7742  &  Everest (1998)     \\
    \bottomrule
  \end{tabular*}
\end{minipage} 
\hfill 
\begin{minipage}{0.49\textwidth}
  \begin{tabular*}{\textwidth}{ r l l l }
    \toprule
    \emph{\#} & \emph{rating} & \emph{title}\\
    \midrule
    1 &  3.8694  &  Apt Pupil (1998)            \\
    2 &  3.4805  &  The Wedding Singer (1998)   \\
    3 &  3.1314  &  Fallen (1998)               \\
    4 &  3.1225  &  Tainted (1998)              \\
    5 &  2.9442  &  Blues Brothers 2000 (1998)  \\
    6 &  2.9046  &  Sphere (1998)               \\
    7 &  2.8842  &  Desperate Measures (1998)   \\
    8 &  2.8798  &  Firestorm (1998)            \\
    9 &  2.8633  &  Vermin (1998)               \\
    10&  2.8511  &  The Prophecy II (1998)      \\
    \bottomrule
  \end{tabular*}
  \end{minipage} 
  \vspace{1em}
  \caption[Ranking Many Results]{
    Ranking many results: 
    In this example, the user search for the query "1998", to get movies from that year.
    The top 10 of these are shown in the left table. As this query matches a lot of 
    movies, the IR method returns a large number of results. By setting the IR weight to $0$,
    and letting the stacked user models do the ranking, the top 10 results change completely,
    while still being good matches for the current query.
  }
  \label{table:rank:year}
\end{table}
