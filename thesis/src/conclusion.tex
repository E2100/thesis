\label{chap:discussion}

This chapter will discuss the implications of our results.
While our hypotheses may be answered,
it is important to clarify what we have actually found out,
and what limits there is to this knowledge.
We will also summarise the contributions of this paper,
and suggest possible future work.


\section{Implications}      

Our central position is that modern aggregate recommender systems 
are constrained by misplaced subjectivity.
Each system selects some measures to model its users, while this selection should be left to each user.
This problem extends to the items recommended by each system: 
Different modeling methods will suit each item differently.

Stacked recommenders can help solve this problem.
In a collection of possible recommender algorithms, each is adaptively used based 
on how well it performs for the item and user in question.
We think the experiments in the previous chapter shows the promise of this technique.
However, there are lots of use cases not yet considered.

It should be clear that stacked recommenders would work best in situations where
we have a wide range of diverse algorithms that can infer the relevance of an item to a user.
For users, social connections is a good example: whether or not social connections should influence
recommendations or personalized search results is a contentious topic.
Naturally, a system where each user's personal opinion determines if these connections are used is desireable.

This implication extens to the items that should be recommended:
As evident by the field of information retrieval,
there exists many ways of considering the relevance of an item. 
Each of these algorithms can be based on a number of attributes:
temporality, geography, sentiment analysis, topic or key words.
It is not a huge leap to consider that each of these algorithms may have
varying levels of accuracy for each individual item.
Stacked recommenders can help solve this problem by adaptively 
combining the recommenders based on individual item performance.

With stacked recommenders, both the methods layer and the adaptive layer consists of standard recommender algorithms.
Because we use ratings matrices for the taste models and error matrices for the weight estimations,
we can use the same algorithms for both tasks.
Using known algorithms for this new task is beneficial:
they are known to work, enjoy multiple implementations
and are already understood and battle-tested in many different systems.

However, as this approach is more complicated than standard recommenders,
it is worth questioning if its gains are worth the extra complexity.
This depends on the basic recommenders that are to be combined.
If the system is made up by many different recommenders,
that each user might place varying importance on,
and that may have varying success with each item,
stacked recommenders may provide implressive gains in accuracy.
On the other hand, if the recommenders are simple in nature,
and look at similar patterns in the data,
generalized aggregation methods might be more applicable.




\section{Contributions} 

We have made two main contributions with this paper:
(1) described the latent subjectitivty problem and
(2) developed the technique of stacked recommenders.

(1) The latent subjectivity problem is one we think hinders
standard recommender systems reaching their full potential.
As far as we know, this problem has not been described
in the context of recommender systems.
The main choice for any such system is how to predict unknown ratings.
To do this, a pattern in the available ratings data must be leveraged.
These patterns are plentiful, and which works best is dependent on
each user and item in the system.
Modern aggregation recommenders utilize many patterns, but on a generalized
level, where each user and item is treated the same.
This underlying subjectivity leads to a mismatch between the notions
of whoever developed the systems, and the users and items of the service.

The latent subjectivity problem extens to any ensemble learning systems
(as those described in \cite{Polikar2006}) that blends multiple 
algorithms to leverage patterns.
Whenever we have multiple algorithms that work on a set of items
(and possibly users), there is a question of how accurate each
approach will be for each item.
Averaged or generalized weighted approaches will always
chose the combination that performs best \emph{on average},
with little concern to the uniqueness of items (and users).
In other words, this is a comprehensive problem
that may be discovered amongst many machine learning techniques.

(2) Stacked recommenders is our attempt to solve the latent subjectivity problem.
As far as we know, this type of adaptive prediction aggregation has not been done before.
Chapter \ref{chap:results} showed that an aggregation that combines predictions based
on estimated accuracy can outperform both standard recommenders and simple aggregation approaches.
Our technique is strengthened by the fact that standard recommender algorithms
are used for the accuracy estimations.
This is the core insight of this paper: 
by creating error models for each recommender, we can use this to predict
its accuracy for each user/item combination.
These predictions can then be used to weigh each combined algorithm accordingly.

As far as the latent subjectivity problem extends to any ensemble learning system,
the adaptive aggregation part of stacked recommenders can be used to 
create better combinations of many types of predictors.
Whenever we have a set of algorithms producing a set of predicted values
based on items, a set of aggregating recommenders can model the probable
errors of these approaches, based on each individual item.
This leads to adaptive ensembles that should outperform generalized approaches.
Because of this, the technique build in this paper should be 
applicable in situations other than recommender systems.

While the experiments of Chapter \ref{chap:results} show the general viability of stacked reocmmenders,
we belive there are greater opportunities in systems where there  are even more diverging
patterns to be leveraged. The prime examples of this are systems that may or may 
not use social connections between users, and systems which predict the 
relevance of widely varying items.


\section{Future Work}      

We have only shown the basic viability of stacked recommenders,
and how they can outperform traditional approaches on traditional datasets.
This section outlines a few interesting research topics
which should shed more light on the subject.

\subsection{Choosing Different Adaptive Recommenders}

We chose to use SVD-based recommenders for the adaptive part of our stacked approach.
The main reason for this is that we are looking for global traits of the data
when performing accuracy estimations. In other words, we wish to identify
clusters of users and items for which each algorithm may or may not be suited.

However, as the stacked recommenders can utilize any standard recommender system
to model the errors of another recommender, it would be interesting to perform
a more in-depth study of how different choices for the adaptive layer
influence the final system.
There are many more recommenders that also look at global patterns
that might be well suited for this task.

Another interesting question is whether other machine learning methods can be used for the adaptive layer.
For example, using neural networks to estimate non linear aggregation functions for each user would be an interesting approach.
This was attempted earlier in our research, but abandoned when recommenders were found to produce
better results in a more elegant way. 


\subsection{Using Stacked Recommenders in Other Domains}

We chose to use the MovieLens dataset and the RMSE evaluation measure for testing our approach.
The primary reason was to be able to direcly evaluate our results towards those of other research papers.
As this dataset and this error measure is widely used to evaluate recommender systems,
it is natural for a first look at a new approach to use the same notions of accuracy.

However, as mentioned above, the main strength of stacked recommenders may be
in situations with much more diverse data sources. Social networks or systems
with widely varying sets of items would provide an interesting use case for stacked recommenders.
The main premise of our approach is that each user and item have differing preferences
for each algorithm. 

Naturally, the more diverse the data and algorithms get,
the more dire the need for adaptive aggregation becomes.
Because of this, using stacked recommeners in other domains with more variation 
in the data and combined algorithms would be an interesting topic.


\subsection{Using Stacked Recommenders in Other AI Fields}

We have only considered the notion of latent subjectivity within the field of recommender systems.
However, as briefly mentioned above, the technique should be applicable to many more situations.
Whenever there is a set of prediction algorithms that use different data to produce results,
an adaptive aggregation should be able to combine these in a more nuanced way.

Ensemble learning is a big topic, used in many situations.
By stacking recommenders on top of each method in an ensemble, 
we get a system capable of predicting the accuracy of each method.
Naturally, it would be interesting to see how this approach would fare
in other fields such as document classification, document clustering,
curve fitting \cite[p7]{Polikar2006}, and other fields of ensemble learning.


\section{Conclusion}

The information overload problem will always be present.
No matter how elegant solutions one may find,
the fact is that the overwhelming amount of available data
quickly outgrows our ability to use it.
We believe artificial intelligence is crucial to finding a solution.
Only by creating intelligent systems that 
help us filter, sort and consume information can we hope 
to mitigate the overload.

This paper has explained the \emph{latent subjectivity problem},
and introduced the technique of \emph{stacked recommenders}
in an attempt to solve it.
We believe this is a real problem,
and that our solution would be a worthy addition to
any system where recommenders play an important role.
Our technique implicitly, and without any extra work required from each user
adapts how the system models users and items based on past performance.
Our experiments show that this technique is capable of higher accuracy
than standard recommenders and simple aggregation approaches.

On a more general note, we think our notion of adaptive user model
aggregation is key to stopping information overload.
Generalized methods is not enough: only
by creating truly adaptive systems that adapt their
algorithms to ecah user and item can we achieve a 
system powerful enough to deal with the widely
varying users, and vast scope of items.

Information is indeed a curious thing,
and our only way of taming the never ending torrent of 
arriving data is to embrace the wide scope of the information,
and the people who wish to consume it.
Adaptive user modeling takes us one step further towards this goal.

