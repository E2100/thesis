\label{chap:discussion}

This chapter will discuss the implications of our results.
While our hypotheses may be answered,
it is important to clarify what we have actually found out,
and what limits there is to this knowledge.
We will also summarise the contributions of this paper,
and suggest possible future work.


\section{Implications}      

Our central position is that modern aggregate recommender systems fail because of misplaced subjectivity.
Each system selects some measures to model its users, while this selection should be left to each user.
This problem extends to the items recommended by each system: 
Different modeling methods will suit each item differently.

Stacked recommenders can help solve this problem.
In a collection of possible recommender algorithms, each is adaptively used based 
on how well it performs for the item and user in question.
We think the experiments in the previous chapter shows the promise of this technique.
However, there are lots of use cases not yet considered.

It should be clear that stacked recommenders would work best in situations where
we have a wide range of diverse algorithms that can infer the relevance of an item to a user.
For users, social connections is a good example: whether or not social connections should influence
recommendations or personalized search results is a contentious topic.
Naturally, a system where each user's personal opinion determines if these connections are used is desireable.

This implication extens to the items that should be recommended:
As evident by the field of information retrieval,
there exists many ways of considering the relevance of an item. 
Each of these algorithms can be based on a number of attributes:
temporality, geography, sentiment analysis, topic or key words.
It is not a huge leap to consider that each of these algorithms may have
varying levels of accuracy for each individual item.
Stacked recommenders can help solve this problem by adaptively 
combining the recommenders based on individual item performance.

With stacked recommenders, both the methods layer and the adaptive layer consists of standard recommender algorithms.
Because we use ratings matrices for the taste models and error matrices for the weight estimations,
we can use the same algorithms for both tasks.
Using known algorithms for this new task is beneficial:
they are known to work, enjoy multiple implementations
and are already understood and battle-tested in many different systems.

However, as this approach is more complicated than standard recommenders,
it is worth questioning if its gains are worth the extra complexity.
This depends on the basic recommenders that are to be combined.
If the system is made up by many different recommenders,
that each user might place varying importance on,
and that may have varying success with each item,
stacked recommenders may provide implressive gains in accuracy.
On the other hand, if the recommenders are simple in nature,
and look at similar patterns in the data,
generalized aggregation methods might be more applicable.





\section{Limitations}


\section{Contributions} 


\section{Future Work}      

\subsection{Different adaptive recommenders}
\subsection{Different domains}


\section{Conclusion}      


