\section{User Meta Modeling}

We define \emph{Meta Modeling} (MM) as the act of using a modeling method to adapt other modeling methods.
Similarily, \emph{User Meta Modeling} (UMM) is then the act of performing user modeling to adapt other user modeling methods.
More specifically, as we wish to combine methods in UM on a personalized level, we need two levels of user modeling:

\textbf{Level 1} (L1) are the conventional, single-approach user modeling methods described in Section \ref{sec:recommender}.
Each of these methods predict unknown ratings between users and items.
This prediction is made based on some pattern of the available data,
e.g. user rating correlations or social connections.  

\textbf{Level 2} (L2) is the aggregator modeling method. This method performs aggregation of results from methods in Level 1,
just as the methods described in Section \ref{sec:aggregate}, but on a personalized level. 
This level combined the predicted ratings from Level 1 into a single prediction, based on the current user.
We shall develop one method applicable to Level 2: A user modeling method that combines multiple predictors into a single
predictions, based on the observed performance of each predictor in relation to the user in question.

Formally, a system for UMM can be described as a 6-tuple:

\begin{eqnarray*}
  \mathrm{UMM} &=& (Items, Users, Ratings, Framework, Methods, Aggregation)\\
               &=& (I,U,R,F,M,A).
\end{eqnarray*}

We have a set of $Items$ and a set of $Users$.
There is also a set of $Ratings$: each user $u \in U$ can \emph{rate} an item $i \in I$.
As before, we use the term "rating" loosely --- other applicable and equivalent terms include \emph{relevance}, \emph{utility},
\emph{connection strength} or \emph{ranking}. In other words, this is a measure of what a user thinks of an item
in the current domain language. However, since "rating" will match the case study we present later in this chapter,
that is what we shall use.

As we are dealing with multiple approaches to user modeling, we have a set of $Methods$ that each create their own
user models. 
This corresponds to Level 1. 
Each model $m \in M$ are used to compute predictions, i.e. estimations of unknown ratings.
As demonstrated in Chapter \ref{chap:theory}, there are many different forms of user modeling,
that each consider differents aspects of the available data: the users, items and ratings, as well as 
other sources such as intra-user connections in social networks or intra-item connections in information retrieval systems.

The $Aggregation$ part of this quintuple refers to how the predictions from the different methods are combined
into one blended prediction. 
This corresponds to Level 2.
To achieve the best possible compounded result, we wish to use methods that look at disjoint patterns, 
i.e. complementary predictive parts of the data.
As found by \citet[p6]{Bell2007} the accuracy of the combined predictor is more dependent on the 
ability of the various predictors to expose different aspects of the data, than on 
the individual accuracy of each predictor.


...


\begin{equation*}
 R_{u,i} =
 \begin{pmatrix}
  r_{1,1} & r_{1,2} & \cdots & r_{1,i} \\
  r_{2,1} & r_{2,2} & \cdots & r_{2,i} \\
  \vdots  & \vdots  & \ddots & \vdots  \\
  r_{u,1} & r_{u,2} & \cdots & r_{u,i}
 \end{pmatrix}
\end{equation*}

Here, $\hat{r}_{ui}$ is the predicted rating from user $u$ of item $i$.
$w_m$ is the weight of method $m$ and $p_m(u,i)$ is the predicted rating from method $m$.
Of course, the weights of each method have to be estimated in advance.
By creating a training and a testing set from already known ratings, we can estimate the weights
that minimize the error over the testing set. 
This can for example be done through \emph{regression} or the \emph{least squares method}.
(TODO: Refs)

User-weighted average

\begin{equation*}
  \hat{r}_{ui} = \sum_{m \in M} w_{um} \cdot p_m(u,i)
\end{equation*}


Higher-order

\begin{equation*}
  \hat{r}_{ui} = f_{u}[ P_{M}(u,i) ]
\end{equation*}


\subsection{Multilayer Perceptron}
\subsection{Modeling Phase}
\subsection{Prediction Phase}




