\section{User Meta Modeling}

So, how do we perform personalized method aggregation?
Let us first define a few terms.
We define \emph{Meta Modeling} (MM) as using one modeling method to adapt other modeling methods.
More specifically, \emph{User Meta Modeling} (UMM) means adapting user modeling methods with another user modeling method.
This leaves us with two distinct levels of user modeling methods: the \emph{methods level} and the \emph{aggregation level}.
Formally, a system for UMM can be described as a 6-tuple:

\begin{eqnarray*}
  \mathrm{UMM} &=& (Items, Users, Ratings, Framework, Methods, Aggregation)\\
               &=& (I,U,R,F,M,A).
\end{eqnarray*}

We have a set of $Items$ and a set of $Users$.
There is also a set of $Ratings$: each user $u \in U$ can \emph{rate} an item $i \in I$.
As before, we use the term "rating" loosely --- other applicable and equivalent terms include \emph{relevance}, \emph{utility},
\emph{connection strength} or \emph{ranking}. In other words, this is a measure of what a user thinks of an item
in the current domain language. However, since "rating" will match the case study we present later in this chapter,
that is what we shall use. 

The $Framework$ variable specifies how this data is represented.
The two canonical ways of representing users, items and ratings are graphs and matrices (see Section \ref{sec:recommender}).
We shall use a matrix, where the first dimension corresponds to users, the second to items, and each populated cell is an explicit rating:

\begin{equation*}
 R_{u,i} =
 \begin{pmatrix}
  r_{1,1} & r_{1,2} & \cdots & r_{1,i} \\
  r_{2,1} & r_{2,2} & \cdots & r_{2,i} \\
  \vdots  & \vdots  & \ddots & \vdots  \\
  r_{u,1} & r_{u,2} & \cdots & r_{u,i}
 \end{pmatrix}
\end{equation*}

As we are dealing with multiple approaches to user modeling, we have a set of $Methods$ that each create their own
user models. 
This corresponds to the \emph{methods layer}.
Each model $m \in M$ are used to compute predictions, i.e. estimations of unknown ratings.
As demonstrated in Chapter \ref{chap:theory}, there are many different forms of user modeling,
that each consider differents aspects of the available data: the users, items and ratings, as well as 
other sources such as intra-user connections in social networks or intra-item connections in information retrieval systems.
Examples include Slope One, SVD and Nearest Neighbor weighted predictions
(see Section \ref{subsec:recommender:examples}).
These methods predict unkown connections between users and items based on some pattern in the data,
for example user correlations or social connections.

The $Aggregation$ part of our 6-tuple refers to how the predictions from the different methods are combined
into one blended prediction. 
This corresponds to the \emph{aggregation level}.
To achieve the best possible compounded result, we wish to use methods that look at disjoint patterns, 
i.e. complementary predictive parts of the data (see Section \ref{sec:aggregate}).
As found by \citet[p6]{Bell2007} the accuracy of the combined predictor is more dependent on the 
ability of the various predictors to expose different aspects of the data, than on 
the individual accuracy of each predictor.
As described in Section \ref{sec:aggregate}, multiple prediction results are normally 
combined into a final singular result,
based on a generalized combination found by minimizing some error across all users.

Another way of describing (and implementing) the two modeling levels is through application
of the $\mathrm{map}$ and $\mathrm{reduce}$ functions of functional programming.
When performing \emph{prediction aggregation} (scores), this estimation can be expressed as

\begin{equation*}
  \hat{r}_{ui} = \mathrm{reduce}(u, \mathrm{map}(M,u,i)).
\end{equation*}

First, each modeling method is applied through the $\mathrm{map}$ function, with the current user and rating for which
a rating should be estimated. This produces a set of scalar prediction values. These values are then
combined through the $\mathrm{reduce}$ method, which takes the predictions and current user as input.
In our case, this is the personalized aggregation method. 
If we wish to do rank aggregation (i.e. sorted lists), the equation is a bit different:

\begin{equation*}
  \tau_{u,n} = \mathrm{reduce}_{n}(u, \mathrm{map}(M_{n},u)).
\end{equation*}

Here, $\tau_{u}$ is the list of recommended items for user $u$ (following the notation in \citet[p3]{Dwork2001}).
Note that there is no input item in this formula as we wish to produce a ranking of the top $n$ recommended items.
Expressing ourselves in terms of $\mathrm{map}$ and $\mathrm{reduce}$ now is helpful, as this will later
guide our implementation of these operations in a proper MapReduce framework
for parallell computation (as explained in \citet[p75]{Manning2008}).


\subsection{Personalized Aggregation}

To perform UMM, we need the $Aggregation$ (or $\mathrm{reduce}$) method to be an actual user modeling method,
that adapts the blending with respect to each user.
Let us now look at how this can be achieved.
Until now we have talked about both prediction aggregation (scores) and rank aggregation (sorted lists).
For now we shall stick to scalar predictions, but will return to rank aggregation in Section \ref{sec:methods:rank}.

The simplest generalized way of prediction aggregation is a simple avereage over all predictions made
by the different methods:

\begin{equation*}
  \hat{r}_{ui} = \frac{1}{N} \sum_{m \in M} p_m(u,i).
\end{equation*}

$\hat{r}_{ui}$ is the estimated rating from user $u$ to item $i$,
$N$ is the number of methods in $M$, and $p_m(u,i)$ is the predicted rating from method $m$.
However, in most cases we wish to weight each method differently:

\begin{equation*}
  \hat{r}_{ui} = \sum_{m \in M} w_{m} \cdot p_m(u,i) \quad \text{where} \quad 0 \leq w \leq 1, \quad \sum_{i \in M} (w_i) = 1.
\end{equation*}

$w_m$ is the weight applied to modeling method $m$. These weights fall in the range $[0,1]$ and sum up to $1$.
As described in \ref{sec:aggregate}, these weights can be estimated using a host of machine learning methods.
The known rating data is separated into a training- and testing set, which is used to estimate optimal weights
by minimizing some error across the testing set 
(note that the training set is also split in some way into two sets to create the singular modeling methods).
However, as discussed in Section \ref{sec:reasoning},
this is still a generalized, averaged result across every user. 
The system assumes that the best average result is the best result for each individual user.

So, we wish the weights to be user-specific. The simplest approach is to create secondary user models for aggregation.
Intuitively, a user-specific weight vector could fill the role of this user model.

\begin{equation*}
  \hat{r}_{ui} = \sum_{m \in M} w_{um} \cdot p_m(u,i)
\end{equation*}

$w_{um}$ is the user specific weight for method $m$. In other words, $w_{u}$ is the user model vector.
Each user then has a pesonal set of weights describing how much they prefer each modeling method.
These weights can be estimated much in the same way as generalized weights, 
with the same training and testing set, just on a per-user basis.

However, a personalized linear combination of modeling methods is not enough.
Because of the disjoint nature of the differing modeling methods, 
they essentially each find some reason that an item should receive a certain score.
Consider the different patterns recommender systems and information retrieval measures might leverage:

\begin{itemize*}
  \item Ratings from other users weighted by user similarity.
  \item The similarity of items through the use of an IR method.
  \item Clustering items and users through global effects, finding general trends.
  \item Ratings based on local averages, generosity and other individual features.
  \item Comparing user preference terms with items.
\end{itemize*}

While a linear combination of such disjoint patterns may achieve accurate scores,
a nonlinear combination should be able to discover less apparent user preferences.
For example, in any case where the agreement of two or more methods are more telling than their contribution
to a linear combination, a nonlinear combination can catch such features of the user's preferences.

Consider the task of sorting incoming emails based on user preferences: 
a user might consider a new mail important if the sender is a social connection
and the last communication from this person happened a long time ago. 
In this case, the agreement between two scoring methods is more important
than any other prediction.

Another case is when one method is especially important for some 
range of its possible predictions. For example, if one method deals with catching possible spam,
a confident capture from this method would probably be more important than 
what any of the other predictions.

In short, we want an aggregation method capable of capturing nonlinear preferences
when it comes to combining user modeling methods.
Let us express this as an equation:

\begin{equation*}
  \hat{r}_{ui} = f_{u}(M(u,i))
\end{equation*}

The user model is the nonlinear function $f_u$ that takes the results of each modeling method $M(u,i)$,
and combines them into a combined prediction.




\subsection{Multilayer Perceptron}
\subsection{Modeling Phase}
\subsection{Prediction Phase}




