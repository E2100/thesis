\section{User Modeling}
\label{sec:modeling}

The term \emph{user modeling} (\smallcaps{UM}) lacks a strict definition. 
Broadly speaking, when an application is adapted in some way based on what the system knows about its users, we have user modeling. 
From predictive modeling methods in machine learning, 
to how interface design is influenced by personalization --- the field covers a lot of ground. 

It is important to differentiate between adapting the interface of an application and the content of an application. 
Many user modeling methods strive to personalize the interface itself, e.g. menus, buttons and control elements 
(e.g. \cite{Jameson2009, Fischer2001}. 
Adapting the application content, on the other hand, means changing how and what content is displayed.
For instance, interface adaption might mean changing the order of items in a menu, while content 
adaption might mean changing the order and emphasis of results in a web search interface
(e.g. \cite{Xu2008, Qiu2006, Rhodes2000}).

In this thesis, we are interested in adapting the \emph{content} of an application.
We believe the information overload problem often stems from a mismatch between presented content and desired content. 
Examples of adaptive content include:

\begin{itemize*}
  \item Suggesting interesting items based on previous activity.
  \item Reorganizing or filtering content based on predicted user relevance.
  \item Translating content based on a user's geographical location.
  \item Changing the presentation of content to match personal preferences or abilities.
  \item Personalizing search results based on previous queries and clicks.
\end{itemize*}

The fields of Artificial Intelligence (AI) and Human-Computer Interaction (HCI) share a common goal solving information overload through user modeling. 
However, as described by \cite[p6]{Lieberman2009}, they have different approaches and their efforts are seldom combined. 
While AI researchers often view contributions from HCI as trivial cosmetics, the HCI camp
tends to view AI as unreliable and unpredictable --- surefire aspects of poor interaction design.

In AI, user modeling refers to algorithms and methods that infer knowledge about a user based on past interaction 
(e.g. \cite{Pazzani2007, Smyth2007, Alshamri2008, Resnick1994}).
By examining previous actions, predictions can be made of how the user will react to future information. This new knowledge is then embedded in a model of the user, which can predict future actions and reactions. 
For instance, an individual user model may predict how interesting an unseen article will be to a user, based on previous feedback on similar articles or the feedback of similar users.

HCI aims to meet user demands for interaction. 
User modeling plays a crucial role in this task. 
Unlike the formal user modeling methods of AI, user models in HCI are often cognitive approximations, manually developed by researchers to describe different types of users 
(e.g. \cite{Fischer2001, Jameson2009, Cato2001}).
These models are then utilized by interaction designers to properly design the computer interface based on a models predictions of its user's preferences.
\cite{Totterdell1990} describes user modeling in interaction design as a collection of deferred parameters: 
\emph{``The designer defers some of the design parameters such that they can be selected or fixed by features of the environment at the time of interaction [...] Conventional systems are special cases of adaptive systems in which the parameters have been pre-set.''}

This thesis is concerned with the AI approach to user modeling, and in particular, the use of \emph{recommender systems} (RSs).
As our goal is to adaptively combine different RSs based on each user and item,
we shall now describe what makes a recommender system, and introduce some of the many algorithms they employ.


