\section{Evaluation Strategies}

To evaluate how our model performs, we need a measure
for computing the total error across a large number of predictions.
The canonical measure for estimating the error of a 
predictions from a recommender system
is the \emph{Root mean squared error} (RMSE) measure
(e.g. \citet[p17]{Herlocker2004}, \citet[p13]{Adomavicius2005} and \citet[p6]{Bell2007}).
We shall use this measure to estimate the performance
of our adaptive prediction aggregation algorithms.
The RMSE of a set of estimations $\hat{R}$, 
compared to a set of known ratings $R$, is defined as

\begin{equation*}
  RMSE(\hat{R},R) = \sqrt{E((\hat{R} - R)^2)}
  = \sqrt{\frac{
      \sum_{i=1}^{n} (\hat{R}_i - R_i)^2
    }{
      n
    }},
\end{equation*}

where $n$ is the total number of predictions.
The RMSE combines a set of errors into one single combined error.
A beneficial feature of the RMSE is that the resulting error 
will be on the same scale as the estimations. In other words,
if we are predicting values on the scale $1-5$, the computed error
will be on this scale as well. In this case, an error of $1$
would then say that we are on average $1$ point away from the true 
ratings on our $1-5$ scale.

While the RMSE works well for evaluating scalar predictions,
we need another measure for considering a predicted sorting 
from rank estimation methods.
Here, we are not interested in the predicted scores,
but rather in which position each item appears in a sorted list of results.
This is for instance needed when measuring the performance of a
personalized search engine.
Because of this, we are interested in examining how 
personalization with stacked recommenders affects the rankings
from an IR method.

To answer our three hypotheses, we have performed two experiments.
The first experiment evaluates our method when used for
adaptive prediction aggregation, comparing it to 
the methods given in Table \ref{table:results:methods}.
This will help us answer hypotheses H1 and H2.
The second experiment will evaluate our performance
in adaptive rank aggregation, in order to answer hypothesis H3.
For more information on how the algorithms
in this experiment were implemented, see Appendix \ref{appendix:implementation}.



