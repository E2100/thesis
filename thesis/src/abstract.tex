\null\vspace{4em}
{
  \centering
  \normalfont
  \huge
  \noindent
  Abstract\\
}
\vspace{2em}

\noindent
In the field of artificial intelligence,
\emph{recommender systems} are methods
for predicting the relevance items to users.
The items can be just about anything, for example 
documents, articles, movies, music, events or other users.
Recommender systems examine data such as ratings, query logs,
user behavior and social connections to predict
what each user will think of each item.

Modern recommender systems
combine multiple standard recommenders
in order to leverage disjoint patterns in available data.
By combining different methods,
complex predictions that rely on much evidence can be made.
These aggregations can for example be done 
by estimating weights that result in an optimal combination.

However, we posit these systems have an important weakness.
There exists an underlying, misplaced subjectivity to relevance prediction.
Each chosen recommender system reflects one view of 
how users and items \emph{should} be modeled.
We believe the selection of recommender methods should 
be automatically chosen based on their predicted accuracy for each user and item.
After all, a system that insists on being adaptive
in one particular way is not really adaptive at all.

This thesis presents a novel method for prediction aggregation
that we call \emph{adaptive recommenders}.
Multiple recommender systems are combined on a per-user and per-item basis
by estimating their individual accuracy in the current context.
This is done by creating a secondary set of error estimating recommenders.
The core insight is that standard recommenders can be used
to estimate the accuracy of other recommenders.
As far as we know, this type of adaptive prediction aggregation
has not been done before.

Prediction aggregation (combining scores) is tested in a recommendation scenario.
Rank aggregation (sorting results lists) is tested in a personalized search scenario.
Our initial results are promising and show that adaptive recommenders
can outperform both standard recommenders and simple aggregation methods.
We will also discuss the implications and limitations of our results.

\cleardoublepage
