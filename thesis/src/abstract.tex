\null\vspace{3em}
{
  \centering
  \normalfont
  \huge
  Abstract\\
}
\vspace{2em}

\noindent
In the field of artificial intelligence,
\emph{recommender systems} are user modeling methods
that can predict the relevance of an item to a user.
Items can be just about anything: documents, articles, movies, music, events,
or indeed, other users.
These powerful systems examines data such as ratings, query logs,
user behaviour and social connections to predict
what each user will think of each available item.

Modern recommender systems blend multiple standard recommenders
in order to leverage disjoint patterns in the available data.
By combining different types of recommenders,
complex predictions that rely on much evidence can be made.
These aggregations are done on a generalized level,
often by weighting each recommender in a way
that achieves an optimal result.

However, we posit these systems have an important weakness.
There exists an underlying, misplaced subjectivity to relevance prediction.
Each chosen recommender system reflects how the system
generally thinks each user and item should be modeled.
We believe the selection of recommender methods should 
be automatically and contextually based on how well they perform for each user and item.

This paper presents a novel method for adaptive prediction aggregation,
called \emph{stacked recommenders}.
Multiple recommender systems are combined on a per-user and per-item basis
by estimating how accurate each recommender will be for the current user and item.
This is done by creating a set of secondary error estimating recommenders.
By estimating predictions and their accuracy based 
on the current user and item,
optimally adaptive predictions can be estimated.
As far as we know, this type of adaptive recommender aggregation
has not been done before.

Prediction aggreation is tested in a recommendations scenario,
and rank aggregation in a personalized search scenario.
Our initial results are promising, showing that stacked recommenders
can outperform both standard recommenders and simple aggregation methods.

\clearpage
