\section{Latent Subjectivity}
\label{sec:reasoning}

As we saw in Chapter \ref{chap:theory}, 
there are lots of ways of predicting the
relevance of an item to a user. 
In fact, judging by the number of different approaches,
the only limiting factor seems to be the different 
patterns researchers discover in available data.
As described in Section \ref{sec:aggregate},
aggregate modeling is a common way of combining different, complimenting
methods into one prediction system.
By leveraging so called \emph{disjoint patterns}
in the data, several less than optimal predictors
can be combined, so that the combination outperforms each part.

However, both simple predictions and aggregate predictions have a fundamental problem.
There exists an underlying, misplaced subjectivity to relevance prediction that is seldom discussed.
When a method is developed or selected for use in a prediction system,
a concious descision of which approach to use is made.
The researcher or developer select which method (or methods in the case of aggregation),
they think should be able to model users of the system.
Consider the following relevance judgements:

\begin{itemize*}
  \item PageRank \citep{Bender2005} assumes that the relevance of a web page is 
  represented by its authority, as computed from inbound links from other sites.
  \item When providing personalized search results, one ranking signal may be 
  the social connections of the current user. Items deemed relevant by the user's 
  peers will then recieve a boosted ranking (e.g. \cite{Carmel2009}).
  %\item When determining the relevance of an email, one predictor might be based
  %on how often the sender sends emails to the current user.
  \item When recommending movies, one predictor may be based on the ratings
  of users with similar profile details. Another predictor might be 
  dependent on some feature, e.g. production year of well liked movies.
  \item Recommendations based on the Pearson Coefficient \cite[p11]{Segaran2007}
  assumes that the statistical correlation between user ratings defines their 
  similarity.
\end{itemize*}

Are these metrics subjective? 
While the methods themselves may perform well, their selection
reflects how whoever created the system assumes how each user
can and \emph{should} be modeled. This underlying, \emph{latent subjectivity} is not desirable.

For example, While one user might appreciate a social
influence in their search results, another user might not.
While one user might find frequency of communication maps well to relevance,
another might not. 
One user might feel the similarity of movie titles are a good predictor,
while another might be more influenced by production year.
The exact differences are not important --- what is important is that they exist.

Another way of explaining latent subjectivity is that 
\emph{user modeling methods are dependent on the subjective assumptions of their creators}.
In other words, each modeling method use some aspect of available data to make predictions,
and the importance of each of these methods is determined by whoever creates the system,
or by the on average best error-minimizing combination, not by each individual user.

Aggregate modeling methods face the same problem of misplaced subjectivity: 
Aggregation is done on a generalized, global level,
where each user is expected to place the same importance on each modeling method.
While the aggregation is of course selected to minimize some error over a testing set,
the subjective nature remains: The compiled aggregation is a generalization,
treating all users the same --- hardly a goal of user modeling.

We propose a method where these descisions are left to each user,
providing an extra level of abstraction and personalization.
This leaves the subjective nature of modeling method selection where it should be:
In the hands of each individual user.
If each method is \emph{only used} based on how well it performs for each user,
any possibly applicable user modeling method suddenly becomes a worthy addition.
Consider the following two questions:

\begin{enumerate*}
  \item What combination of which methods will accurately predict unknown scores?
  \item Which methods could possibly help predict a score for a user?
\end{enumerate*}

The first question is what has to be considered in traditional modeling aggregation:
First a set of applicable methods leveraging disjoint patterns must be selected. 
Then, an optimal and generalized combination of these must be found,
most often through minimizing the average error across all users.

The second question is quite different. 
Instead of looking for an optimal set of methods and an optimal combination,
we look for the set of \emph{any applicable method} that \emph{some users} might find helpful.
We believe this is a much simpler problem: 
instead of trying to generalize individuality,
it should be embraced, by allowing users to implicitly and automatically select which methods they prefer,
from a large set of possible predictors.

The issue of latent subjectivity can also cause problems when considering the scope of items.
Should we really use the same metrics for judging every item's relevance score?
Needless to say, items are often quite different from another,
along a myriad of dimensions. Consider the World Wide Web:
If each webpage is an item, the number of metrics we can use to judge
the relevance of an item is immense.
If items are indeed as different as the users themselves, it stands to reason that the same 
modeling method will not perform as well for every item.
We are left with the same two questions as before:

\begin{enumerate*}
  \item What combination of which methods will accurately model every item?
  \item Which methods could possibly help predict a score for an item?
\end{enumerate*}

As before, an approach where we only need to consider the second question is desireable.
Regardless, both traditional, single-approach modeling methods, and modern aggregation approaches
often treat every item the same. No matter its intrinsic qualities, an item will be judged
by the same methods as every other item. 

This chapter will develop a way to aggregate a host of modeling methods on a per-user and per-item basis.
By adapting the aggregation to the current item and user, we sidestep the issue of
latent subjectivity. The user is in control of which methods best fit their needs, and
each method's priority is influenced by how well it performs for the current item.
We will now express our goals as a three hypotheses.


\clearpage
