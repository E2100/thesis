% This file has been generated from the lua sources using LuaDoc.
% To regenerate it call "make genluadoc" in
% doc/generic/pgf/version-for-luatex/en.

\paragraph{pgflibrarygraphdrawing-interface.lua}


\begin{luacommand}{{Interface:addEdge}(\meta{from},\meta{to},\meta{direction},\meta{options})}
Adds an edge from one node to another by name.  That is, both parameters are node names and have to exist before an edge can be created between them.

Parameters:
\begin{itemize}
	\item[] \meta{options} \subitem A key=value string, which is currently only passed back to the \TeX layer during shipout (deprecated, use \tikzname\ keys instead).
\end{itemize}



See also:
\begin{itemize}
	\item[] |addNode|
\end{itemize}

\end{luacommand}\begin{luacommand}{{Interface:addNode}(\meta{name},\meta{xMin},\meta{yMin},\meta{xMax},\meta{yMax},\meta{options})}
Adds a new node to the graph.  The options string is parsed and assigned.

Parameters:
\begin{itemize}
	\item[] \meta{name} \subitem Name of the node.\item[] \meta{xMin} \subitem Minimum x point of the bouding box.\item[] \meta{yMin} \subitem Minimum y point of the bouding box.\item[] \meta{xMax} \subitem Maximum x point of the bouding box.\item[] \meta{yMax} \subitem Maximum y point of the bouding box.\item[] \meta{options} \subitem Options to pass to the node (deprecated, use \tikzname\ keys instead).
\end{itemize}



\end{luacommand}\begin{luacommand}{{Interface:drawEdge}(\meta{object})}
Helper function to put visible edges back to the TeX layer.

Parameters:
\begin{itemize}
	\item[] \meta{object} \subitem Lua edge object to draw.
\end{itemize}



\end{luacommand}\begin{luacommand}{{Interface:drawGraph}()}
Draws/layouts the current graph using the specified algorithm.  The algorithm is derived from the options attribute and is loaded on demand from the corresponding file, e.g. for algorithm ``simple'' it is ``pgflibrarygraphdrawing-algorithms-simple.lua'' which has to define a function named ``drawGraphAlgorithm\_simple'' in the pgf.graphdrawing module.  It is then called with the graph as single parameter.



\end{luacommand}\begin{luacommand}{{Interface:drawNode}(\meta{object})}
Helper function to actually put the node back to the TeX layer.

Parameters:
\begin{itemize}
	\item[] \meta{object} \subitem The lua node object to draw.
\end{itemize}



\end{luacommand}\begin{luacommand}{{Interface:finishGraph}()}
Pops the top graph from the graph stack (which is the current graph) and actually draws the nodes and edges on the canvas.



\end{luacommand}\begin{luacommand}{{Interface:getOption}(\meta{name})}
Returns the value of the graph option name.

Parameters:
\begin{itemize}
	\item[] \meta{name} \subitem Name of the option.
\end{itemize}


Return value:
\begin{itemize} \item[] The stored value or nil. \end{itemize}


\end{luacommand}\begin{luacommand}{{Interface:loadAlgorithm}(\meta{name})}
Loads the file with the ``pgflibrarygraphdrawing-algorithms-xyz.lua'' naming scheme.

Parameters:
\begin{itemize}
	\item[] \meta{name} \subitem Name of  the algorithm, like ``xyz''.
\end{itemize}


Return value:
\begin{itemize} \item[] The algorithm function or nil. \end{itemize}


\end{luacommand}\begin{luacommand}{{Interface:newGraph}(\meta{options})}
Creates a new graph and pushes it on top of the graph stack.  The options string is parsed and assigned.

Parameters:
\begin{itemize}
	\item[] \meta{options} \subitem A list of options for this graph (deprecated, use \tikzname\ keys instead).
\end{itemize}



See also:
\begin{itemize}
	\item[] |finishGraph|
\end{itemize}

\end{luacommand}\begin{luacommand}{{Interface:setOption}(\meta{name},\meta{value})}
Sets a graph option name to value.

Parameters:
\begin{itemize}
	\item[] \meta{name} \subitem The name of the option to set.\item[] \meta{value} \subitem New value for the option.
\end{itemize}



\end{luacommand}
