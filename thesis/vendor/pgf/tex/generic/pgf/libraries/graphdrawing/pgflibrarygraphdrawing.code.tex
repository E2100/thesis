% Copyright 2010 by Renée Ahrens, Olof Frahm, Jens Kluttig, Matthias Schulz, Stephan Schuster
%
% This file may be distributed and/or modified
%
% 1. under the LaTeX Project Public License and/or
% 2. under the GNU Public License.
%
% See the file doc/generic/pgf/licenses/LICENSE for more details.

\ProvidesFileRCS[v\pgfversion] $Header: /cvsroot-fuse/pgf/pgf/generic/pgf/libraries/graphdrawing/pgflibrarygraphdrawing.code.tex,v 1.1 2011/04/13 19:56:44 matthiasschulz Exp $

%\catcode`\@11\relax

% check if luatex is running
\ifx\directlua\relax{%
  \errmessage{You need LuaTeX to use the graph drawing library} 
  }
\else\ifx\directlua\undefined{%
    \errmessage{You need LuaTeX to use the graph drawing library}
\fi\fi

%
% This file defines the basic interactions between LUA and PGF
%-----------------------------------------------------------------

%
% This box is used for passing pgf elements to lua and vice versa 
%
% After the invocation of \pgfpositionnodelater the caller should copy the contents of 
% \box\pgfpositionnodelaterbox to another box register (for this refer to the manual).
%
\newbox\pgf@gd@box

% 
% Initialize the lua graph drawing environment
%
\directlua{
  local file = 'pgflibrarygraphdrawing-loader.lua'
  local format = 'tex'
  % Use either resolvers or kpse to locate files.
  if resolvers then
    dofile(assert(resolvers.find_file(file, format), "couldn't find file " .. file))
  else
    dofile(assert(kpse.find_file(file, format), "couldn't find file " .. file))
  end
  % set the box number for saving nodes by pgf
  pgf.graphdrawing.Sys:setBoxNumber(\the\pgf@gd@box)
}


%
% Begins a new graph drawing scope
%
% Usage
%   \pgfgdbeginscope
%
\def\pgfgdbeginscope{%
  \pgf@gd@logmessage{GD:SYS: scope began}
  \pgf@gd@beginscope%
}

%
% Ends a graph drawing scope
%
% This macro invokes the selected graph drawing algorithm and
% ships out all nodes within this scope
%
% See \pgfgdbeginscope
%
\def\pgfgdendscope{%
  \pgf@gd@logmessage{GD:SYS: scope ended}%
  \directlua{
    pgf.graphdrawing.Interface:drawGraph()
    pgf.graphdrawing.Interface:finishGraph()
  }}

%
% Adds an edge to the graph
%
% #1 left anchor
% #2 right anchor
%
% Usage
%   \pgfgdaddedge{node1}{node2}{direction}
%  or
%   \pgfgdaddedge[options]{node1}{node2}{direction}
%
% where options can be a combination of every possible tikz options (deprecated,
% options should be stored using the TikZ key mechanism)
%
% Example
%
%   \pgfgdaddedge{NodeFrom}{NodeTo}{->}
%
\def\pgfgdaddedge{%
  \pgfutil@ifnextchar[%
    {\pgf@gd@addedge}
    {\pgf@gd@addedge[]}
}

%
% Begins the ship out of nodes
%
\def\pgfgdbeginshipout{%
  \pgf@gd@beginshipout
}

%
% End the shop out of nodes
%
\def\pgfgdendshipout{%
  \pgf@gd@endshipout
}


%
% Enables the verbose logging of the graph drawing library
%
\def\pgfgdenableverboselogging{
  \pgf@gd@set@verbose@mode1%
}

%
% Disables the verbose logging of the graph drawing library
\def\pgfgddisableverboselogging{
  \pgf@gd@set@verbose@mode0%
}

%
% Prints a given message to the TeX output
%
% Note: logging must be enabled by \pgfgdenableverboselogging
%
% Example:
%
%  \pgfgdenableverboselogging
%  \pgfgdlogmessage{Hello world}
%
\def\pgfgdlogmessage#1{
  \pgf@gd@logmessage#1%
}

%
% INTERNAL MACROS
%-------------------------------------------------------------------------------
%

%
% This macro delays the node placement on the pgf layer
%
\def\pgf@gd@positionnodelater{%
  \pgf@gd@logmessage{GD:SYS: positionnodelater invoked}
  \pgfpositionnodelater{\pgf@gd@positionnode@callback}}

%
% Callback method for \pgf@gd@positionnodelater
% 
% Pipes the box which contains the nodes to lua.
% The called lua method then saves the data.
%
\def\pgf@gd@positionnode@callback{%
  \pgf@gd@logmessage{GD:SYS: positionnode callback invoked for box \the\pgf@gd@box}
  {%
    % save options to macro \pgf@gd@node@options
    \pgfkeysgetvalue{/tikz/graphs/graph drawing/@node@options}\pgf@gd@node@options
    % save pgfpositionnodelaterbox (see manual)
    \setbox\pgf@gd@box=\box\pgfpositionnodelaterbox\relax
    % call lua system library to create a lua node object
    \directlua{
      %% NOTE: options parameter has to be the key=value string
      pgf.graphdrawing.Interface:addNode(
        "\luaescapestring{\pgfpositionnodelatername}",
        "\luaescapestring{\pgfpositionnodelaterminx}",
        "\luaescapestring{\pgfpositionnodelaterminy}",
        "\luaescapestring{\pgfpositionnodelatermaxx}",
        "\luaescapestring{\pgfpositionnodelatermaxy}",
        "\luaescapestring{\pgf@gd@node@options}") }
    % clear node options
    \pgfkeyslet{/tikz/graphs/graph drawing/@options}\pgfutil@empty
  }%
}

%
% Begins a new graph drawing scope
%
\def\pgf@gd@beginscope{%
  \pgf@gd@logmessage{GD:SYS: begin scope}%
  % get options
  \pgfkeysgetvalue{/tikz/graphs/graph drawing/@options}\pgf@gd@graph@options%
  \pgf@gd@logmessage{GD:SYS: key options: \pgf@gd@graph@options}
  \directlua{
    pgf.graphdrawing.Interface:newGraph("\luaescapestring{\pgf@gd@graph@options}")
  }%
  \pgf@gd@positionnodelater
}

%
% Shipout a node
%
% #1 = name of the node
% #2 = x min of the bounding box
% #3 = x max of the bounding box
% #4 = y min of the bounding box
% #5 = y max of the bounding box
% #6 = desired x pos of the node
% #7 = desired y pos of the node
% #8 = box register number of the TeX node
%
% This is an internal function and will be called by the Sys-class of the Lua part
%
% Example
%
%  \pgfgdinternalshipoutnode{not yet positionedPGFGDINTERNALnodename}{10}{10}{20}{20}{10pt}{10pt}0
% 
\def\pgfgdinternalshipoutnode#1#2#3#4#5#6#7#8{%
  {%
    \pgf@gd@logmessage{GD:TEX: positioning node #1 (#2,#3,#4,#5) to #6,#7 from register #8}\relax
    \def\pgfpositionnodelatername{#1}
    \def\pgfpositionnodelaterminx{#2}
    \def\pgfpositionnodelatermaxx{#3}
    \def\pgfpositionnodelaterminy{#4}
    \def\pgfpositionnodelatermaxy{#5}
    \setbox\pgf@gd@box=\box\pgfutil@voidb@x
    \directlua{
      texnode = pgf.graphdrawing.TeXBoxRegister:getBox(#8)
      assert(texnode,"GD:SYS:TEX: tex node was nil")
      tex.box[\the\pgf@gd@box] = texnode
    }
    \setbox\pgfpositionnodelaterbox=\box\pgf@gd@box
    \pgfpositionnodenow{\pgfqpoint{#6pt}{#7pt}}
  }%
}

%
% Begins a new pgf/TikZ scope for shipping out nodes
%
% The scope permits other nodes referencing graph drawing nodes.
%
\def\pgf@gd@beginshipout{%
  \scope
}

%
% Ends a shipout
%
\def\pgf@gd@endshipout{%
  \endscope
}

%
% Adds an edge to the lua graph object
%
% #1 first node
% #2 second node
% #3 edge options
%
% Example
%
%  \pgf@gd@addedge{NodeFrom}{NodeTo}{->}
\def\pgf@gd@addedge[#1]#2#3#4{%
  \directlua{
    pgf.graphdrawing.Interface:addEdge("\luaescapestring{#2}","\luaescapestring{#3}","\luaescapestring{#4}","\luaescapestring{#1}")
  }}


%
% LOGGING
% ------------------------------------------------------------------------------
%

%
% New if for debugging mode
%
\newif\ifpgf@gd@verbose

% 
% Set verbose debugging mode
%
% #1 should be zero (0) or one (1) for false and true
%
\def\pgf@gd@set@verbose@mode#1{
  \ifodd#1
    \directlua{ pgf.graphdrawing.Sys:setVerboseMode(true) }
    \pgf@gd@verbosetrue
  \else
    \directlua{ pgf.graphdrawing.Sys:setVerboseMode(false) }
    \pgf@gd@verbosefalse
  \fi
}

%
% Logs a message to the console (tex output, log file)
%
% #1 string to print out to console
%
\def\pgf@gd@logmessage#1{\ifpgf@gd@verbose\directlua{ texio.write_nl("\luaescapestring{#1}") }\fi}
